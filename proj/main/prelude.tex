\usepackage{listings}
\usepackage{xcolor}
 
\definecolor{codegreen}{rgb}{0,0.6,0}
\definecolor{codegray}{rgb}{0.5,0.5,0.5}
\definecolor{codepurple}{rgb}{0.58,0,0.82}
\definecolor{backcolour}{rgb}{0.95,0.95,0.92}
 
\lstdefinestyle{mystyle}{
    backgroundcolor=\color{backcolour},   
    commentstyle=\color{codegreen},
    keywordstyle=\color{magenta},
    numberstyle=\tiny\color{codegray},
    stringstyle=\color{codepurple},
    basicstyle=\ttfamily\footnotesize,
    breakatwhitespace=false,         
    breaklines=true,                 
    captionpos=b,                    
    keepspaces=true,                 
    numbers=left,                    
    numbersep=5pt,                  
    showspaces=false,                
    showstringspaces=false,
    showtabs=false,                  
    tabsize=2
}
 
\lstset{style=mystyle}

%packages
\newcommand{\mg}[1]{\textcolor[rgb]{.90,0.00,0.00}{[MG: #1]}}
\newcommand{\gp}[1]{\textcolor[rgb]{0.00,0.5,0.5}{[GP: #1]}}
\usepackage{geometry}
\usepackage{multicol}
\geometry{left=2.7cm,right=2.7cm, top=2cm,bottom=3.2cm}
\usepackage{blindtext}
\usepackage{subcaption}
\usepackage{caption}

\usepackage{tabu}

\usepackage{natbib}
\usepackage{mathtools}
\usepackage{mdframed}
\usepackage{booktabs}
% \usepackage{hyperref}
\usepackage{siunitx} % Provides the \SI{}{} and \si{} command for typesetting SI units
\usepackage{graphicx} % Required for the inclusion of images
% \usepackage{natbib} % Required to change bibliography style to APA
\usepackage{datetime}
\usepackage{lscape}
\usepackage{algorithm}
\usepackage{algorithmic}
\usepackage{xspace}
\usepackage[english]{babel} % English language/hyphenation
\usepackage{proof}
\usepackage{booktabs} % Top and bottom rules for tables
\usepackage[colorlinks, allcolors = blue,]{hyperref}
\usepackage{accents}
\usepackage{amsfonts}
\usepackage{stmaryrd}
\SetSymbolFont{stmry}{bold}{U}{stmry}{b}{n}
\usepackage{amsmath,amsthm,amssymb,latexsym} 
\usepackage{microtype}
\usepackage{graphicx}
% \usepackage{subfigure}
\usepackage{booktabs} % for professional tables
\usepackage{hyperref}
\usepackage{lipsum}

\usepackage{authblk}


%new commands
\newcommand{\theHalgorithm}{\arabic{algorithm}}
% \newtheorem{definition}{Definition}
\usepackage{cancel}
\usepackage[normalem]{ulem}
\newcommand{\dataobs}{\boldsymbol{x}}
\newcommand{\dataX}{\boldsymbol{X}}
\newcommand{\datay}{\boldsymbol{y}}
\newcommand{\dataz}{\boldsymbol{z}}
\newcommand{\adj}[2]{\boldsymbol{adj}(#1,#2)}
\newcommand{\candidateset}[1]{\mathcal{R}_{#1}}
\newcommand{\bprior}{\boldsymbol{\beta}_{\textup{prior}}}
\newcommand{\bysinfer}[1]{\mathsf{DirP}(#1)}
\newcommand{\bys}[2]{\mathsf{DirP}(#1,#2)}

\newcommand{\bysdir}{\mathsf{DIR}}
\newcommand{\betad}{\mathsf{Beta}}
\newcommand{\betaf}{\textup{B}}
\newcommand{\mbetaf}{\boldsymbol{\textup{B}}}
\newcommand{\vtheta}{\boldsymbol{\theta}}
\newcommand{\valpha}{\boldsymbol{\alpha}}
\newcommand{\vbeta}{\boldsymbol{\beta}}
\newcommand{\vLambda}{\boldsymbol{\Lambda}}
\newcommand{\vmu}{\boldsymbol{\mu}}
\newcommand{\lapmech}{\mathsf{LSDim}}
\newcommand{\ilapmech}{\mathsf{LSHist}}
\newcommand{\binomial}[2]{\mathsf{Bin}(#1, #2)}
\newcommand{\multinomial}[2]{\mathsf{Mult}(#1, #2)}
\newcommand{\expmech}{\mathsf{EHD}}
\newcommand{\hexpmech}{\mathsf{EHDS}}
\newcommand{\lexpmech}{\mathsf{EHDL}}
\newcommand{\hexpmechd}{\mathsf{expMech}^{D}_{\hellinger}}
\newcommand{\privinfer}{\mathsf{PrivInfer}}
\newcommand{\hlg}{\mathsf{H}}
\newcommand{\dirichlet}[1]{\mathsf{Dir}(#1)}
\newcommand{\inferd}[3]{\mathsf{DirP}(#1,#2,#3)}
\newcommand{\dirf}[4]{\mathsf{Dir}_{#2, #3}(#1,#4)}
\newcommand{\alphas}{\boldsymbol{\alpha}}
\newcommand{\xis}{\boldsymbol{\xi}}
\newcommand{\iverson}[1]{[#1]}
\newcommand{\datauni}{\mathcal{X}}
\newcommand{\hellinger}{\mathcal{H}}
\newcommand{\ux}[1]{u(\textbf{x}, {#1})}
\newcommand{\uxadj}[1]{u(\textbf{x}', {#1})}
\newcommand{\cardinality}[2]{\mathcal{C}^{#1}_{#2}}
\newcommand{\range}{\mathcal{O}}
\newcommand{\nomalizer}[1]{\sum\limits_{r'\in \mathcal{R}_{\textup{post}}} \exp \big(\frac{-\epsilon\cdot \mathcal{H} (\mathsf{BI}(#1),r')}{4 \cdot S(#1)}\big)}

\newcommand{\unomalizer}[1]{\sum\limits_{r'\in \mathcal{R}_{\textup{post}}} \exp \big(\frac{-\epsilon\cdot u(#1, r')}{4 \cdot S(#1)}\big)}


\newcommand{\hexpmechPr}[2]{\underset{z \thicksim \hexpmech(#1)}{\Pr}\left[ #2 \right]}
\newcommand{\lapmechPr}[2]{\underset{z \thicksim \lapmech(#1)}{\Pr}\left[ #2 \right]}

\newcommand{\ilapmechPr}[2]{\underset{
{z \thicksim \ilapmech(#1)}
}{\Pr}\left[ #2 \right]}

\newcommand{\normdis}{\mathcal{N}}

\newtheorem{thm}{Theorem}[section]
\newtheorem{lem}{Lemma}[section]
\theoremstyle{definition}
\newtheorem{defn}{Definition}[section]
% \newtheorem{def}{Definition}[section]
\newtheorem{assert}{Assertion}[lem]
\newcommand{\lap}[2]{\mathsf{Lap}(#1, #2)}
\newcommand{\todo}[1]{{\footnotesize \color{red}\textbf{[[ #1 ]]}}}