\documentclass[11pt]{article}
\usepackage[margin=1in,footskip=0.25in]{geometry}
\usepackage{hyperref}
\usepackage{graphicx}
\usepackage[T1]{fontenc}
\usepackage{listings}
\usepackage{xcolor}
 
\definecolor{codegreen}{rgb}{0,0.6,0}
\definecolor{codegray}{rgb}{0.5,0.5,0.5}
\definecolor{codepurple}{rgb}{0.58,0,0.82}
\definecolor{backcolour}{rgb}{0.95,0.95,0.92}
 
\lstdefinestyle{mystyle}{
    backgroundcolor=\color{backcolour},   
    commentstyle=\color{codegreen},
    keywordstyle=\color{magenta},
    numberstyle=\tiny\color{codegray},
    stringstyle=\color{codepurple},
    basicstyle=\ttfamily\footnotesize,
    breakatwhitespace=false,         
    breaklines=true,                 
    captionpos=b,                    
    keepspaces=true,                 
    numbers=left,                    
    numbersep=5pt,                  
    showspaces=false,                
    showstringspaces=false,
    showtabs=false,                  
    tabsize=2
}
 
\lstset{style=mystyle}

\usepackage{mathtools}

\DeclarePairedDelimiter\abs{\lvert}{\rvert}%
\DeclarePairedDelimiter\norm{\lVert}{\rVert}%

% Swap the definition of \abs* and \norm*, so that \abs
% and \norm resizes the size of the brackets, and the 
% starred version does not.
\makeatletter
\let\oldabs\abs
\def\abs{\@ifstar{\oldabs}{\oldabs*}}
%
\let\oldnorm\norm
\def\norm{\@ifstar{\oldnorm}{\oldnorm*}}
\makeatother

\newcommand*{\Value}{\frac{1}{2}x^2}

\newcommand{\caseL}[1]{\item[\textbf{case}] \textbf{#1}\newline}
\newcommand{\subcaseL}[1]{\item[\textbf{subcase}] \textbf{#1}\newline}

\newcommand{\todo}[1]{{\footnotesize \color{red}\textbf{[[ #1 ]]}}}


\newcommand{\data}{\mathbf{x}}

\newcommand{\datay}{\mathbf{y}}
\newcommand{\domain}{\mathcal{X}}
\newcommand{\domainy}{\mathcal{Y}}


\usepackage{amsthm}
\usepackage{amsmath}
\usepackage{graphicx}
\usepackage{multicol, latexsym, amssymb}
\usepackage{blindtext}
\usepackage{subcaption}
\usepackage{caption}
\usepackage{algorithm}
\usepackage{algorithmic}

\usepackage{tabu}

\begin{document}

\title{
{\textbf{CS591S1 Homework 1: Attacks and DP Basics}}
}
\author{Jiawen Liu\\
Collaborators: none.}

\date{}
\maketitle

\section{Problem 1}

\begin{enumerate}
	\item The only input to the attacker is a (the vector of noisy counters).
\begin{itemize}
\item \textbf{Algorithm Description}
%
%
\begin{algorithm}
\caption{Reconstruction Attack without Extra Information}
\label{alg_p1-1}
\begin{algorithmic}
\REQUIRE The observed results $a$ from query.
\STATE {\bf Initialize $c = a$}, as the reconstructed counter . 
\STATE  {\bf for}\ $i\in [a.length]$\ {\bf do}.  
\STATE \qquad {\bf let} 
	$s = a[i] - c[i - 1]$ 
	\COMMENT {be the noised $i_{th}$ item in data set.}
\STATE \qquad {\bf If} $s > 1$  {\bf do}.
		$c[i] = a[i] - 1$
		\COMMENT {the correct counter at position i must be a[i] - 1.}
\STATE \qquad {\bf Elif} $s < 0$  {\bf do}.
\STATE \qquad \qquad 
		{\bf let} $j = i$
\STATE \qquad \qquad {\bf While} 
		$j - 1 > 0 \land c[j] < c[j - 1]$  {\bf do}.
		$j = j - 1$;
		$c[j] = c[j] - 1$
\STATE	\COMMENT {the correct counter at position i must be a[i] and decrease all the previous abnormal items by 1}

\RETURN $s_i = c_{i} - c_{i - 1}$ {\bf for}\ $i\in [a.length]$.
\end{algorithmic}
\end{algorithm}
%
\item \textbf{Experimental Results Plotting}
%
Figure \ref{fig-p1-1}
%
\begin{figure}[h]
    \centering
    \includegraphics[width=0.6\textwidth]{p1-1.eps}
    \caption{P1 - 1 : accuracy of recontruction attack without aux information}
    \label{fig-p1-1}
\end{figure}
%
\item \textbf{Results Discussion}
%

%
The average accuracy is 0.7775 from the Alg. \ref{alg_p1-1}, which is higher than 0.75.
%
\\%
I also got different results form some attempts before I achieved this result:
%
\\
%
\textbf{attempt 1.}
I first tried the naive method in Alg. \ref{alg_p1-1-2}, which only achieved standard 0.75 accuracy on average.
%
\begin{algorithm}
\caption{Naive Reconstruction Attack without Extra Information}
\label{alg_p1-1-2}
\begin{algorithmic}
\REQUIRE The observed results $a$ from query.
\STATE {\bf Initialize vector s: $s[i] = a[i] - a[i - 1]$} 
\COMMENT {as the reconstructed dataset.} 
\STATE  {\bf for}\ $i\in [a.length]$\ {\bf do}.  
\STATE \qquad {\bf If} $s > 1$  {\bf do}.
    $s[i] = 1$
\STATE \qquad {\bf Elif} $s < 0$  {\bf do}.
	$s[i] = 0$
\RETURN $s$.
\end{algorithmic}
\end{algorithm}
%
\\
%
\textbf{Attempt 2.}
Then I tried the minimizing error method from class in Alg. \ref{alg_p1-1-3}, which achieved a better accuracy but is inefficient. 
I'm able to achieve higher accuracy for small size data base with large iteration number $T$.
However,
I cannot even get the results for $n = 1000$ in resonable time.
%
\begin{algorithm}
\caption{Minimizing Error Reconstruction Attack without Extra Information}
\label{alg_p1-1-3}
\begin{algorithmic}
\REQUIRE The observed results $a$ from query, the number of iteration $T$, the query result release mechanism $M$.
\STATE {\bf Initialize dataset s, error e, $s[i] \leftarrow \{0, 1\}$, $e = Inf$}. 
\STATE  {\bf for}\ $i\in [T]$\ {\bf do}.  
\STATE \qquad sample a new data set $s'$
\STATE \qquad $e' = sum(a - M(s'))$;
\STATE \qquad  {\bf If}\ $ e' < e$\ {\bf do}. s = s'; e = e'
\RETURN $s$.
\end{algorithmic}
\end{algorithm}
%
\\
%
\textbf{Attempt 3, Alg. \ref{alg_p1-1}.}
%
Then I go back to the Alg. \ref{alg_p1-1-2} and made some improvements in it.
\begin{itemize}
	\item Instead of recovering the data base directly, I firstly recovered the counter.
	% 
	\item In the situation where $a[i] - a[i - 1] < 0$, instead of just setting $s[i]$ be $0$, I went back and retraced the previous result.
\end{itemize}
These two steps improved the accuracy from the naive method.

\item \textbf{Code Documentation}
%
See Appendix \ref{code-p1-1}.
%
\item{Repository Link}

\url{ https://github.com/jiawenliu/CS591S1.git }
\end{itemize}

\item The attacker's inputs consist of a and the vector of guesses w.

	\begin{itemize}
	\item \textbf{Algorithm Description}
	%
	\begin{algorithm}
	\caption{Reconstruction Attack with Extra Information}
	\label{alg_p1-1}
	\begin{algorithmic}
	\REQUIRE The observed results $a$ from query, the guass $w$.
	\STATE {\bf Initialize $c = a$}, as the reconstructed counter. 
	\STATE {\bf Initialize $s' = [-1] * n$}, as the reconstructed data set where $n$ is the size of the dataset. 
	\STATE  {\bf for}\ $i\in [a.length]$\ {\bf do}.  
	\STATE \qquad {\bf let} 
		$s' = a[i] - c[i - 1]$ be the possible $j_{th}$ item in data set.
	\STATE \qquad {\bf If} $s' > 1$  {\bf do}.
			$c[i] = a[i] - 1$
	\STATE \qquad \qquad 
			{\bf let} $j = i$; $s[i] = 1$
	\STATE \qquad \qquad {\bf While} 
			$j - 1 \geq 0 \land c[j] > c[j - 1]$  {\bf do}.
			$j -= 1$
			$s[j] = 1$
			\COMMENT {the correct counter at position i must be a[i] - 1}
	\STATE \qquad {\bf Elif} $s' < 0$  {\bf do}.
	\STATE \qquad \qquad 
			{\bf let} $j = i$; $s[i] = 0$
	\STATE \qquad \qquad {\bf While} 
			$j - 1 > 0 \land c[j] < c[j - 1]$  {\bf do}.
			$j -= 1$
			$c[j] -= 1$
	\STATE	\COMMENT {the correct counter at position i must be a[i] and decrease the previous by 1 until their difference $\geq 0$}
	\STATE  {\bf for}\ $i\in [a.length]$\ {\bf do}.  
	\STATE \qquad {\bf If} $s[i] == -1$  {\bf do}.
			$s[i] = w[i]$
	\STATE	\COMMENT {If $s[i]$ is -1, means s[i] is uncertain, before we are using c[i] - c[i-1], right now we are using guess[i], which has higher accuracy}

	\RETURN $s$.
	\end{algorithmic}
	\end{algorithm}
	%
	\item \textbf{Experimental Results Plotting}
	%
	Figure \ref{fig-p1-2}
\begin{figure}[h]
    \centering
    \includegraphics[width=0.6\textwidth]{p1-2.eps}
    \caption{P1 - 2 : accuracy of recontruction attack with aux information}
    \label{fig-p1-2}
\end{figure}	%
	\item \textbf{Results Discussion}
	%
	This one has higher accuracy than without using side information around $0.82$.

	Because when without side information, in the case of uncertainty, we can only guess without any help. The accuracy of random guess is $\frac{1}{2}$. But when we have the guess of accuracy $\frac{2}{3}$, we guess higher accuracy in the case of uncertainty. Then the accuracy will be improved.
	%
	\item \textbf{Code Documentation}
	%
	%
	The Code are exactly the same as code in previous part, except adding one function in Appendix \ref{code-p1-2}
	%
\end{itemize}
\end{enumerate}

\section{Problem 2}
\begin{enumerate}
\item
\begin{enumerate}
	\item[\textbf{(a)}]
	Given $X_i(j)$ sampled from $\{-1, 1\}$ i.i.d for $i = 1, \cdots, n$ and $j = 1, \cdots, d$, we have:
	\[
	\begin{array}{c}
	E(X_i(j)) = 0, ~ Var(X_i(j)) = 1\\
	E(\bar{X}(j)) = E(\frac{1}{n}\sum_i X_i(j)) = 0, 
	~ Var(\bar{X}(j)) = Var(\frac{1}{n}\sum_i X_i(j)) = \frac{1}{n}\\
	\end{array}
	\]
	Since $A = M(X) = \bar{X} + (Z(1), \cdots, Z(d))$ where $Z(j) \sim N(0, \sigma^2)$ independently from $X$, we have:
	\[
	E(A(j)) = 0, ~ Var(A(j)) = Var(\bar{X}(j) + Z(j)) = \frac{1}{n} + \sigma^2
	\]
	%
	Since $T_{out}$ is independently from $A$, we can conclude the expectation of $F(A, T_{out})$:
	\[
	E(F(A, T_{out})) = E(\sum_{j = 1}^{d}A(j) * T_{out}(j))
	= \sum_{j = 1}^{d} E(A(j) * T_{out}(j))
	= 0
	\]
	%
	and the standard deviation:
	\[
	\begin{array}{rl}
	& \sqrt{Var(F(A, T_{out}))}\\
	= & \sqrt{Var(\sum_{j = 1}^{d}A(j) * T_{out}(j))}\\
	= & \sqrt{\sum_{j = 1}^{d}Var(A(j) * T_{out}(j))}\\
	= & \sqrt{ \sum_{j = 1}^{d}Var(A(j)) * Var(T_{out}(j))
	+ Var(A(j))*E(T_{out}(j))^2 + E(A(j))^2 * Var(T_{out}(j))}\\
	= & \sqrt{d \big(\frac{1}{n} + \sigma^2\big)}
	\end{array}
	\]
	%
	\item[\textbf{(b)}]
	Since $T_{in}$ is sampled from $X$ which is not independently from $A$ anymore, we have following:
	\[
	\begin{array}{ccl}
	F(A, T_{in}) 
	& = & \sum_{j = 1}^{d}A(j) * T_{in}(j)\\
	& = & \sum_{j = 1}^{d}(\bar{X}(j) + Z(j)) * T_{in}(j)\\
	& = & \sum_{j = 1}^{d}(\frac{1}{n} \sum_{i = 1}^{n}X_i(j) + Z(j)) * T_{in}(j)\\
	& = & \sum_{j = 1}^{d}
	(\frac{1}{n}( \sum\limits_{i = 1\ldots n \land i \neq in}
	X_i(j) + Z(j))* T_{in}(j) + \frac{1}{n}T_{in}(j)^2)
	\end{array}
	\]
	%
	Let $A_{/T_{in}} = \frac{1}{n}( \sum\limits_{i = 1\ldots n \land i \neq in}
	X_i(j) + Z(j))$, we have:
	%
	\[
	E(A_{/T_{in}}) = 0, 
	~ Var(A_{/T_{in}}) = (\frac{n - 1}{n^2} + \sigma^2)
	\]
	%
	So we have the expectation of $F(A, T_{in})$ as:
	%
	\[
	E(F(A, T_{in})) 
	= \sum_{j = 1}^{d}
	E((A_{/T_{in}} * T_{in}(j) + \frac{1}{n}T_{in}(j)^2))
	= \frac{d}{n}
	\]
	%
	and variance of $F(A, T_{in})$ as:
	%
	\[
	\begin{array}{ccl}
	Var(F(A, T_{in})) 
	& = & \sum_{j = 1}^{d}
	Var((A_{/T_{in}} * T_{in}(j) + \frac{1}{n}T_{in}(j)^2))\\
	& = & \sum_{j = 1}^{d}
	Var((A_{/T_{in}} * T_{in}(j))\\
	& = & d(\frac{n - 1}{n^2} + \sigma^2)
	\end{array}
	\]
	%
	The standard deviation is the square root of variance, i.e., $\sqrt{d(\frac{n - 1}{n^2} + \sigma^2)}$.
	%
	\item[\textbf{(c)}]
	\[
	d = 
	\bigg(\sqrt{\big(n + n^2\sigma^2\big)} 
	+ \sqrt{\big(n - 1 + n^2\sigma^2\big)} \bigg)^2
	\]
	%
	\begin{itemize}
		\item $\sigma = 0.1$
		\[
		d = 
		\bigg(\sqrt{\big(n + 0.01n^2\big)} 
		+ \sqrt{\big(n - 1 + 0.01n^2\big)} \bigg)^2
		\]
		%
		\item $\sigma = 1/n$
		\[
		d = 
		\bigg(\sqrt{\big(n + 1\big)} 
		+ \sqrt{\big(n\big)} \bigg)^2
		\]
		%
		\item $\sigma = 1/\sqrt{n}$
		\[
		d = \big(\sqrt{2n} 
		+ \sqrt{2n - 1} \big)^2
		\]
	\end{itemize}
\end{enumerate}

\item
\begin{enumerate}
	\item[\textbf{(a)}]
	%
	\begin{itemize}
		\item \textbf{Algorithms Description}
		%
		The Alg. \ref{alg_p2-2a}
		%
		\begin{algorithm}
		\caption{Inference Attack with Gaussi Mechanism}
		\label{alg_p2-2a}
		\begin{algorithmic}
		\REQUIRE The row and column \# of data base $d, n$, the parameter $\sigma$ for Guassi mechanism.
		\STATE {\bf Sample $d_{in}, d_{out} \leftarrow \{-1, 1\}^{d \times n}$}. 
		\STATE {\bf let query}\ $q(d) = map(sum, d.transpose)/n$
		\STATE {\bf let}\ $A = q(d_{in}) + [z_i \leftarrow N(0, \sigma) for\ i \in [d]]$
		\STATE {\bf let}\ 
		$score_{in}, score_{out} = A \dot d_{in}.transpose, A \dot d_{out}.transpose$

		\STATE	\COMMENT {the F score for each row in dataset in and dataset out.}
		
		\STATE {\bf let}\ 
		$c = $ Counting how many items in $score_{in}$ greater than all items in $score_{out}$.
		
		\STATE {\bf let}\  $p = c / n$

		\RETURN True positive $p$
		\end{algorithmic}
		\end{algorithm}
		%
		\item \textbf{Experimental Results Plotting}
		Figure \ref{fig-p2-2a}

		\begin{figure*}[t!]
		    \centering
		    \begin{subfigure}[t]{0.5\textwidth}
		        \centering
		        \includegraphics[width=\textwidth]{p2-gauss-1.eps}
		        \caption{$\sigma = 0.01$}
		    \end{subfigure}%
		    ~ 
		    \begin{subfigure}[t]{0.5\textwidth}
		        \centering
		        \includegraphics[width=\textwidth]{p2-gauss-2.eps}
		        \caption{$\sigma = 1/3$}
		    \end{subfigure}
		    \caption{P2 : Gaussian mechanism}
		    \label{fig-p2-2a}
		\end{figure*}
%
		\item \textbf{Results Discussion}
		%
		Repeating the Algorithm. \ref{alg_p2-2a}, we get the results poltted in Fig. \ref{fig-p2-2a}. 
		%
		\\
		%
		{\bf The Fig. \ref{fig-p2-2a}(a)} shows the True positive rate with guassi noise parameter $\sigma = 0.01$. We can see the accuracy achieved $1.0$ when data base columns number is $2000$. Actually already achieved $0.92$ when column number greater than $800$.
		%
		\\
		%
		{\bf The Fig. \ref{fig-p2-2a}(b)} shows the True positive rate with guassi noise parameter $\sigma = 1/3$. We can see the accuracy only achieved $0.38$ when data base columns number is $2000$. Even though the column number increase up to $5000$, accuracy only get to around $0.68$.
		%
		\\
		From the two comparison, we can tell the smaller the parameter $\sigma$ is, the more accuracy it is. This is consistent with our intuition that the smaller noise results in more accuracy.

		\item \textbf{Code}
		%
		See Appendix \ref{code-p2-2a}
	\end{itemize}

	\item[\textbf{(b)}]
	%
	\begin{itemize}
		\item \textbf{Algorithms Description}
		%
		The Alg. \ref{alg_p2-2b}
		%
		\begin{algorithm}

		\caption{Inference Attack with Rounding Mechanism}
		\label{alg_p2-2b}
		\begin{algorithmic}
		\REQUIRE The row and column \# of data base $d, n$, the parameter $\sigma$ for Guassi mechanism.
		\STATE {\bf Sample $d_{in}, d_{out} \leftarrow \{-1, 1\}^{d \times n}$}. 
		\STATE {\bf let query}\ $q(d) = map(sum, d.transpose)/n$
		\STATE {\bf let}\ $A = round( q(d_{in}) )$
		\STATE {\bf let}\ 
		$score_{in}, score_{out} = A \dot d_{in}.transpose, A \dot d_{out}.transpose$

		\STATE	\COMMENT {the F score for each row in dataset in and dataset out.}
		
		\STATE {\bf let}\ 
		$c = $ Counting how many items in $score_{in}$ greater than all items in $score_{out}$.
		
		\STATE {\bf let}\  $p = c / n$

		\RETURN True positive $p$
		\end{algorithmic}
		\end{algorithm}

%

		\item \textbf{Experimental Results Plotting}
		%
		Figure \ref{fig-p2-2b}

		\begin{figure*}[t!]
		    \centering
		    \begin{subfigure}[t]{0.5\textwidth}
		        \centering
		        \includegraphics[width=\textwidth]{p2-rounding-1.eps}
		        \caption{$\sigma = 0.01$}
		    \end{subfigure}%
		    ~ 
		    \begin{subfigure}[t]{0.5\textwidth}
		        \centering
		        \includegraphics[width=\textwidth]{p2-rounding-2.eps}
		        \caption{$\sigma = 1/3$}
		    \end{subfigure}
		    \caption{P2 : rounding mechanism}
		    \label{fig-p2-2b}
		\end{figure*}
		%
		\item \textbf{Results Discussion}
		%

		%
		Repeating the Algorithm. \ref{alg_p2-2a} with different column number, we get the results poltted in Fig. \ref{fig-p2-2b}. 
		%
		\\
		%
		{\bf The Fig. \ref{fig-p2-2b}(a)} shows the True positive rate with guassi noise parameter $\sigma = 0.01$. We can see the accuracy achieved $1.0$ when data base columns number is $2000$. Actually already achieved $0.92$ when column number greater than $800$.
		%
		\\
		%
		{\bf The Fig. \ref{fig-p2-2b}(b)} shows the True positive rate with guassi noise parameter $\sigma = 1/3$. We can see the accuracy only achieved $0.9$ when data base columns number is $2000$. However, when the column number increase up to $5000$, accuracy only get to around $1.0$.
		%
		\\
		From the two plots for rounding mechanism, we can tell the smaller the parameter $\sigma$ is, the more accuracy it is. This is consistent with our intuition that the smaller noise results in more accuracy.

		By comparison with the Guassi Mechanism, we can obtain that the in small (i.e. $0.01$) noise parameter, the guassi mechanism can achieve better accuracy. But in larger noise mechanism (i.e. $1/3$) rounding mechanism is able to get a better accuracy.

		\item \textbf{Code}
		See Appendix \ref{code-p2-2b}.

\end{itemize}

\end{enumerate}

\item
\begin{itemize}
	\item \textbf{Algorithms Description}
	Utilizing the library function and simply compare if the products are greater than a threshold, the algorithm shown in Alg. \ref{alg_p2-3}.

		\begin{algorithm}
		\caption{Linear Regression Attack}
		\label{alg_p2-3}
		\begin{algorithmic}
		\REQUIRE The row and column \# of data base $d, n$, the parameter $\sigma$ for Guassi mechanism.
		\STATE {\bf Sample $d, d_{out} \leftarrow \{-1, 1\}^{d \times n}$}. 
		\STATE {\bf Sample $coff \leftarrow \{-1/math.sqrt(d), 1/math.sqrt(d)\}^{d}$}. 
		\STATE {\bf let label}\ $y, y_{out} = genlabel(d, coff), genlabel(d, coff)$.
		
		\STATE {\bf let}\ 
		$classifier(w, target) = 
		1.0\ if\ abs(w \cdot target[0]^T - target[1]) < 0.01\ else\ 0.0$
		
		\STATE {\bf let}\ 
		$score_{in}, score_{out} = classifier(w, d), classifier(w, d_{out})$
		
		\STATE {\bf let}\ 
		$tc = $ Counting \# items in $score_{in}$ is 1, 
		$fc = $ Counting \# items in $score_{out}$ is 0.
		
		\STATE {\bf let}\  $tp = tc / n$, $fp = fc / n$

		\RETURN True positive $tp$ and True negative $fp$
		\end{algorithmic}
		\end{algorithm}

	\item \textbf{Experimental Results Plotting}
	%
	Figure \ref{fig-p2-3}.
%
\begin{figure}[h]
    \centering
    \includegraphics[width=0.5\textwidth]{p2-3.eps}
    \caption{P3: True Positive and True Negtive of Linear Regression Attack}
    \label{fig-p2-3}
\end{figure}
%	
	\item \textbf{Results Discussion}
	Close to perfect.
	\item \textbf{Code}
	See Appendix \ref{code-p2-3}


\end{itemize}
\end{enumerate}


\section{Problem 3}
\begin{enumerate}
	\item (reading)
	\item (Sampling)
	\begin{enumerate}
		\item 
		Given $S_i$ sampled from $a_1, \ldots, a_n$, we have:
		\[
		E(S_i) = \bar{a}.
		\]
		%
		Let $E(S) = E(\sum S_i) = k \bar{a}$, we have:
		%
		\[
		\begin{array}{ccl}
		Pr[error \leq \alpha] 
		& = & Pr[|\hat{S} - \bar{a}| \leq \alpha]\\
		%
		& = & Pr[|\frac{1}{k}\sum S_i - \bar{a}| \leq \alpha] = Pr[|\sum S_i - k\bar{a}| \leq k\alpha]\\
		%
		& = & Pr[|S - E(S)| \leq k\alpha]\\
		%
		& \geq & 1 - 2 \exp \big( 
		- \frac{\alpha^2 / \bar{a}^2}
		{2 + \alpha / \bar{a}} k \bar{a} \big)
		(\text{applying the Chernoff bounds})\\
		\end{array}
		\]
		%
		By guaranteeing $\delta = 2 \exp \big( 
		- \frac{\alpha^2 / \bar{a}^2}
		{2 + \alpha / \bar{a}} k \bar{a} \big)$, we have:
		%
		\[
		k = \frac{\ln(\frac{2}{\delta})(2 + \frac{\alpha}{\bar{a}})\bar{a}}
		{\alpha^2}
		\]
		%
		%
		\item
		let $\mu_i$ be the true proportion of individuals in the population who answer ``yes'' for function $f_i$, i.e. $\frac{1}{n}\sum_{j = 1}^{n}f_i(x_j) = \mu_i$. We know $E[f_i(S_j)] = \mu_i$ and $E[f_i(S)] = k\mu_i$. Let $f_i(S) = \sum_{j = 1}^{k}f_1(S_j)$, we have for all $f_i$:
		%
		\[
		\begin{array}{ccl}
		& & Pr[
		|\frac{1}{k}f_1(S) - \mu_1| 
		\leq \alpha 
		\land \ldots \land 
		|\frac{1}{k}f_d(S) - \mu_d| 
		\leq \alpha]\\
		& = & 1 - Pr[
		|\frac{1}{k}f_1(S) - \mu_1| 
		\geq \alpha  
		\lor \ldots \lor 
		|\frac{1}{k}f_d(S) - \mu_d|
		\geq \alpha] \\
		& = & 1 - Pr[
		|f_1(S) - k\mu_1| 
		\geq k\alpha  
		\lor \ldots \lor 
		|f_d(S) - k \mu_d|
		\geq k\alpha] \\
		%
		& \geq & 1 - \sum\limits_{i = 1}^{d} 
		Pr[|f_1(S) - k \mu_1| \geq k \alpha]
		%
		~~~~(\text{applying the union bound})\\
		%
		& \geq & 1 - 2 \sum\limits_{i = 1}^{d}
		\exp(- \frac{\alpha^2 / \mu_i}{ 2 + \alpha / \mu_i}
		k \mu_i) 
		%
		~~~~(\text{applying the Chernoff bound})\\
		%
		& = & 1 - 2 \sum\limits_{i = 1}^{d}
		\exp(- \frac{k \alpha^2}{ 2\mu_i + \alpha })\\
		%		
		& \geq & 1 - 2d\exp(- \frac{k \alpha^2}{ 2 + \alpha })
		\end{array}
		\] 
		%
		To guaranteeing the $\delta = 2d\exp(- \frac{k \alpha^2}{ 2 + \alpha })$, we have:
		%
		\[
		d = \frac{\ln(2d \frac{1}{\delta})(2 + \alpha)}
		{\alpha^2}
		= \frac{(\ln(2d) + \ln( \frac{1}{\delta}))(2 + \alpha)}
		{\alpha^2}
		= O(\frac{\ln(2d) + \ln( \frac{1}{\delta})}
		{\alpha^2})
		\]
		%
		%
		\item
		Given $err = \frac{Hamming(\hat{x}, x)}{n}$, i.e.
		% 
		$$err = \frac{1}{n}||\hat{x} - x||_{l_1} 
		= \frac{1}{n} \sum_{i = 1}^{n} |\hat{x}(i) - x(i)|$$
		%
		Let $x'$ be the sub data set which is not used to answer mechanism, and $\hat{x'}$ be the gauss of this subset.
		%
		The best the adversary can do is perfectly recover the subset used ($x \setminus x'$) and randomly guess $x'$, so we have:
		\[
		E(err)  = \frac{1}{n}( \frac{2 n}{3} E(|\hat{x'}(i) - x'(i)|) + \frac{n}{3} * 0) = \frac{1}{3}
		\]
		By applying the Chernoff bound, we have:
		\[
		Pr[err \geq (1 + \epsilon) \frac{1}{3}]
		\leq 1 - \exp(- \frac{\epsilon^2}{2 + \epsilon} \frac{n}{3})
		\]
		%
		Given $\Omega(n) \sim \exp(- \frac{\epsilon^2}{2 + \epsilon} \frac{n}{3})$, we have:
		%
		\[
		err \geq (1 + \epsilon) \frac{1}{3} \geq \frac{1}{3} > \frac{1}{4}
		\]

	\end{enumerate}
\end{enumerate}

\section{Problem 4}

\begin{enumerate}
	\item (Sample Size Calculations.)
\begin{enumerate}
	\item
	Given $X_i \sim Bernoulli(p)$, we have:
	\[
	E(X_i) = p, ~~ Var(X_i) = p(1 - p).
	\]
	Since $\bar{X}$ is the mean of i.i.d. $X_1, \cdots, X_n$, we have:
	\[
	E(\bar{X}) = p, ~~ Var(\bar{X}) = \frac{1}{n}p(1 - p).
	\]
	%
	To guarantee that variance of $\bar{X} < \alpha$, the following bound on n can be derived by:
	\[
	\begin{array}{ccc}
	Var(\bar{X}) = \frac{1}{n}p(1 - p)& < & \alpha\\
	n & > & \frac{p(1 - p)}{\alpha} ~ (\star).
	\end{array}
	\]
	%
	Then we have $n_{\epsilon}(\alpha, p)$ be the ceil of $(\star)$:
	\[
	n_{\epsilon}(\alpha, p) = \Bigg\lceil \frac{p(1-p)}{\alpha} \Bigg\rceil.
	\]
	%
	%
	%
	Given $A_{\epsilon}(X) = \bar{X} + Z$ where $Z \sim Lap(1/n\epsilon)$, we have:
	\[
	Var(A_{\epsilon}(X)) = Var(\bar{X}) + Var(Z) = \frac{1}{n}p(1 - p) + \frac{2}{(n\epsilon)^2}
	\]
	%
	Similarly, to guarantee that variance of $\bar{X} < \alpha$, the following bound on $n^*$ can be derived by:
	\[
	\begin{array}{ccc}
	Var(A_{\epsilon}(X)) = 
	\frac{1}{n}p(1 - p) + \frac{2}{(n\epsilon)^2} & < & \alpha\\
	n & > & \frac{p(1 - p)}{\alpha - \frac{2}{(n\epsilon)^2}} ~ (\diamond).
	\end{array}
	\]
	%
	Then we have $n_{\epsilon}(\alpha, p)$ be the ceil of $(\star)$:
	\[
	n^*_{\epsilon}(\alpha, p) 
	= \Bigg\lceil \frac{p(1-p)}{\alpha - \frac{2}{(n\epsilon)^2}} \Bigg\rceil
	\]
	%
	%
	%
	%
	%
	\item 
	%
	\[
	n_{\epsilon}(\alpha, p) - n^*_{\epsilon}(\alpha, p)
	= p(1-p)\big( \frac{1}{\alpha} - \frac{1}{\alpha - 2/(n\epsilon)^2} \big)
	\]
	%
	Since 
	\[
	\big( \frac{1}{\alpha} - \frac{1}{\alpha - 2/(n\epsilon)^2} \big)
	\left \{ 
	\begin{array}{lr}
	> 0 &  \alpha < \frac{1}{\alpha - 2/(n\epsilon)^2}\\
	%
	< 0 & \alpha > \frac{1}{\alpha - 2/(n\epsilon)^2}
	\end{array}
	\right \},
	\] 
	%
	we have the value of $n_{\epsilon}(\alpha, p) - n^*_{\epsilon}(\alpha, p)$:
	%
	\[
	\left \{ 
	\begin{array}{lr}
	\left \{
	\begin{array}{ll}
	\text{decrease} & p \in [0.5, 1]\\
	\text{increase} & p \in [0, 0.5)
	\end{array}
	\right \}
	&  \alpha < \frac{1}{\alpha - 2/(n\epsilon)^2}\\
	%
	\left \{
	\begin{array}{ll}
	\text{increase} & p \in [0.5, 1]\\
	\text{decrease} & p \in [0, 0.5)
	\end{array}
	\right \}
	& \alpha > \frac{1}{\alpha - 2/(n\epsilon)^2}
	\end{array}
	\right \},
	\] 
	
	%
	\item 
	%
	When $A_{\epsilon}$ is the $\epsilon$-DP randomized response estimator, we have:
	%
	\[
	q_{\epsilon}(X_i) = 
	\left\{
	\begin{array}{cc}
	X_i 	& w.p. ~ \frac{e^\epsilon}{1 + e^{\epsilon}} \\
	1 - X_i & w.p. ~ \frac{1}{1 + e^{\epsilon}}
	\end{array}\right \},
	%
	~
	A_{\epsilon}(X) = \frac{1}{n} \sum_{i} q_{\epsilon}(X_i)
	\]
	we have the expectation and variance for $q_{\epsilon}(X_i)$ as:
	\[
	\begin{array}{ll}
	E(q_{\epsilon}(X_i)) 
	& = E(\frac{e^\epsilon}{1 + e^{\epsilon}}X_i 
	+ \frac{1}{1 + e^{\epsilon}}(1 - X_i)) 
	= \frac{1}{1 + e^{\epsilon}}( (e^{\epsilon} - 1)E(X_i) + 1)
	= \frac{1}{1 + e^{\epsilon}}( (e^{\epsilon} - 1)p + 1)\\
	Var(q_{\epsilon}(X_i))
	& = E \big( (q_{\epsilon}(X_i) - E(q_{\epsilon}(X_i)) )^2 \big) 
	= E\big( 
	\frac{e^\epsilon}{1 + e^{\epsilon}}
	\big(X_i - E(q_{\epsilon}(X_i)) \big)^2
	+
	\frac{1}{1 + e^{\epsilon}}
	\big((1 - X_i) - E(q_{\epsilon}(X_i)) \big)^2
	\big)
	\end{array}
	\]
	Let $E_{q_{\epsilon}} = E(q_{\epsilon}(X_i))$, we have:
	\[
	\begin{array}{ll}
	Var(q_{\epsilon}(X_i))
	& = \frac{1}{1 + e^{\epsilon}}
	E((1 + e^{\epsilon})(X_i^2 + E_{q_{\epsilon}}^2)
	+ 1 - 2E_{q_{\epsilon}} + 2(1 - e^{\epsilon})X_iE_{q_{\epsilon}})\\
	%
	& = (p + E_{q_{\epsilon}}^2) + \frac{1}{1 + e^{\epsilon}}(1 - 2E_{q_{\epsilon}} + 2(1 - e^{\epsilon})pE_{q_{\epsilon}})\\
	%
	& = (\frac{1}{1 + e^{\epsilon}} + p - E_{q_{\epsilon}}^2)
	\end{array}
	\]
	Then we have:
	\[
	\begin{array}{ll}
	E(A_{\epsilon}(X))
	& = E\big(
	\frac{1}{n} \sum_{i} q_{\epsilon}(X_i)
	\big) 
	= \frac{1}{1 + e^{\epsilon}}( (e^{\epsilon} - 1)p + 1)\\
	Var(A_{\epsilon}(X))
	& = \frac{1}{n} Var\big( q_{\epsilon}(X_i) \big) \big) 
	= \frac{1}{n}
	(\frac{1}{1 + e^{\epsilon}} + p - E_{q_{\epsilon}}^2)
	\end{array}
	\]
	%
	To guarantee $Var(A_{\epsilon}(X)) > \alpha$, we have:
	\[
	n^*_{\alpha, p} > 
	\frac{\frac{1}{1 + e^{\epsilon}} + p - E_{q_{\epsilon}}^2}
		 {\alpha}
	\]



\end{enumerate}

\item (Global Sensitivities.)
\begin{enumerate}
	\item $1.0$
	\item $1.0$
	\item $\frac{1}{n}$
	\item $1.0$
	\item $(n - 1)$, where $n$ is the size of the Census data (n is not explicitely noted).
	\item $1.0$
	\item $\infty$
	\item $1.0$
\end{enumerate}
\item (Amplification by subsampling)
$B_{p, \epsilon}$ is $\ln(1 + (e^{\epsilon} - 1)p)$-differentially private.
\begin{proof}
Given $A_{\epsilon}$ is $\epsilon$-DP, we have for arbitrary adjacent data set $x$, $x'$ and $G$ be the subset of output domain:
\[
	e^{-\epsilon} \leq 
	\frac{Pr[A_{\epsilon}(x) \in G]}{Pr[A_{\epsilon}(x') \in G]} 
	\leq e^{\epsilon}
\]
So we have for $B_{p, \epsilon}$ under the same setting where $x, x'$ differ in $x_k$ and $S$ is the subset:
\[
\begin{array}{ccc}
	\frac{Pr[B_{p, \epsilon}(x') \in G]}
	{Pr[B_{p, \epsilon}(x) \in G]}
	%
	& = & 
	\frac{Pr[A_{\epsilon}(S) \in G \land x_k \notin S  
	\cup A_{\epsilon}(S') \in G \land x_k \in S]}
	{Pr[A_{\epsilon}(S) \in G]}
	\\
	%
	& \leq &
	\frac{Pr[A_{\epsilon}(S) \in G \land x_k \notin S]
	+ Pr[A_{\epsilon}(S') \in G \land x_k \in S]}
	{Pr[A_{\epsilon}(S) \in G]}
	\\
	%
	& = &
	\frac{(1 - p)Pr[A_{\epsilon}(S) \in G]
	+ p Pr[A_{\epsilon}(S') \in G]}
	{Pr[A_{\epsilon}(S) \in G]}
	\\
	%
	& \leq &
	\frac{(1 - p)Pr[A_{\epsilon}(S) \in G]
	+ p e^{\epsilon}Pr[A_{\epsilon}(S) \in G]}
	{Pr[A_{\epsilon}(S) \in G]}
	\\
	%
	& = & 1 + (e^{\epsilon} - 1)p
\end{array}
\]
The other side can be proved trivially. Then we have $B_{p, \epsilon}$ is $\ln(1 + (e^{\epsilon} - 1)p)$-DP.

\end{proof}
%
%
\item (K-means algorithm)
\begin{itemize}
\item \textbf{Algorithm Description}
%
The algorithm is the same as algorithm presented in class
%
\begin{algorithm}
\caption{Differentially Private K-means}
\label{alg_p4-3}
\begin{algorithmic}
\REQUIRE The rows and columns of the data points $n, d$, the parameters $k, T, \epsilon$, Centers $M$ and parameter $\sigma$ for generating data points.
\STATE {\bf Generate data points $x_i = M + (z \leftarrow N(0, \sigma)^d)$}
\STATE  {\bf let}\ $k = M.length$, $\epsilon' = \frac{\epsilon}{2T}$.  
\STATE  {\bf Initialize}\ $c$ randomly from $[-1, 1]^{d \times k}$. 
\STATE 	{\bf For}\ $t \in [T]$  {\bf do}.
\STATE \qquad	$s_j = \{i : c_j$ is the closet center to $x_i \}$.
\STATE \qquad	$n_j = |s_j| + Lap(\frac{1}{\epsilon'})$ 
\STATE \qquad	$a_j = \sum_{i \in s_j}x_i
			+ Lap(\frac{2}{\epsilon'})^d$ 
\STATE \qquad	$c_j^t = \frac{a_j}{n_j}$ if $n_j > 5$ else randomly from $[-1, 1]^{d \times k}$.
		
\RETURN $c$.
\end{algorithmic}
\end{algorithm}

%
\item \textbf{Experimental Results Plotting}
%
Figrue \ref{fig-p4-3}.
		\begin{figure*}[t!]
		    \centering
		    \begin{subfigure}[t]{0.5\textwidth}
		        \centering
		        \includegraphics[width=\textwidth]{p4-3a.eps}
		        \caption{Pure K-Means}
		    \end{subfigure}%
		    ~ 
		    \begin{subfigure}[t]{0.5\textwidth}
		        \centering
		        \includegraphics[width=\textwidth]{p4-3b.eps}
		        \caption{Private K-Means}
		    \end{subfigure}
		    \caption{P4: K-Means Algorithms}
		    \label{fig-p4-3}
		\end{figure*}
%
\item \textbf{Results Discussion}
%
The results are shown in Fig. \ref{fig-p4-3}. 
%
\\
%
\textbf{In Fig. \ref{fig-p4-3}(a)}, I firstly implemented the pure k-means algorithm which performs very well with only $2$ or even $1$ iteration arrived the true center.
%
\\
%
Then \textbf{in Fig. \ref{fig-p4-3}(b)}, the differentially private k-means algorithm performs also good but with few more iteration in order to arrive the true center.
%
\\
%
During my experiments, some times the algorithm doesn't converge to a final center because of the randomness. 

\item \textbf{Code Documentation}
See Appendix \ref{code-p4-3}
%

\end{itemize}

\end{enumerate}

\section*{Appendix}

\begin{lstlisting}[label=code-p1-1, language=Python, caption=Python Code For Problem 1 - 1, Attack without Side Information]
import random
import numpy as np

#GENERATING DATA SIZE AND CONRRESPONDING PARAMETER
def gen_dataset(n):
	return [random.randint(0, 1)for i in range(n)]
def gen_datasizes(r, step):
	return [i*step for i in range(r[0]/step,r[1]/step + 1)]

#RELEASING THE NOIZED VERSION OF DATABASE
def releasing_dataset(dataset):
	return [sum(dataset[:(i + 1)]) + random.randint(0,1) for i in range(len(dataset))]

#ATTACK WITH ONLY THE KNOWLEDGE OF THE OBSERVATION OF ONE DATABASE
def attack_no_aux(observation):
	rec_counter = [observation[0] - 1 if observation[0] > 1 else observation[0]]
	for i in range(1, len(observation)):
		s = observation[i] - rec_counter[i - 1]
		if s > 1:
			rec_counter.append(observation[i] - 1)
		elif s < 0:
			rec_counter.append(observation[i])
			j = i
			while j - 1 >= 0 and rec_counter[j] < rec_counter[j - 1]:
				j -= 1
				rec_counter[j] -= 1
		else:
			rec_counter.append(observation[i])
	return np.array([rec_counter[0]] + [rec_counter[i] - rec_counter[i - 1] for i in range(1, len(observation))])

###ATTACK OF MINIMIZING ERROR WITH ONLY THE KNOWLEDGE OF OBSERVATION
def attack_no_aux_minerror(observation):
	n, error, r = len(observation), float("inf"), observation
	for i in range(1000):
		s = gen_dataset(n)
		e = sum(abs(releasing_dataset(gen_query(n), s) - observation))
		if e < error:
			r = s
			error = e
	return r

###CALCULATING THE ACCURACY
def accuracy(att, data):
	return sum([1 if att[i]==data[i] else 0 for i in range(len(att))])/(len(att)*1.0)

def plot_accuracy(ys, ns):
	plt.figure()
	plt.plot(ns, ys, "ro-", label = "Accuracy.")
	plt.plot(ns,[sum(ys)/len(ys)]*len(ys), "b-", label="Average Acc.",linewidth=3.0)
	plt.xlabel("n / size of the database")
	plt.ylabel("accuracy / fraction of the bits recovered")
	plt.title("Linear Attack")
	plt.legend()
	plt.grid()
	plt.show()

#EXPERIMENTING WITH FIXED N FOR K ROUNDS
def exprmt_k(n, k):
	acc = 0.0
	for i in range(k):
		dataset, q = gen_dataset(n), gen_query(n)
		acc += accuracy(attack_no_aux(releasing_dataset(q,dataset)),dataset)
	return acc/k
def exprmt_k_ns(ns, k):
	return [testing_kround(n, k) for n in ns]

if __name__ == "__main__":
	datasizes = gen_datasizes((100,900),100)+gen_datasizes((1000,5000),200)+[50000]
	plot_accuracy(exprmt_k_ns(datasizes, 20), datasizes)

\end{lstlisting}


\begin{lstlisting}[label=code-p1-2, language=Python, caption=Python Code For Attack with side Information]
def attack_with_aux(observation, guess):
	rec_counter, n = [observation[0] - 1 if observation[0] > 1 else guess[0]], len(observation)
	certain = [-1] * n

	for i in range(1, n):
		s = observation[i] - rec_counter[i - 1]
		if s > 1:
			rec_counter.append(observation[i] - 1)
			j = i
			certain[j] = 1
			while j - 1 >= 0 and rec_counter[j - 1] < rec_counter[j] and certain[j - 1] == -1:
				j -= 1
		elif s < 0:
			rec_counter.append(observation[i])
			j = i
			certain[j] = 0
			while j - 1 >= 0 and rec_counter[j] < rec_counter[j - 1]:
				j -= 1
				rec_counter[j] -= 1
				certain[j] = 0
		else:
			rec_counter.append(observation[i])
	for i in range(n):
		if certain[i] == -1:
			certain[i] = guess[i]
	return np.array(certain)

def exprmt_k(n, k):
	acc = 0.0
	for i in range(k):
		dataset, q = gen_dataset(n), gen_query(n)
		guess = [d if random.random() >= (1.0/3) else 1 - d for d in dataset]
		acc += accuracy(attack_with_aux(releasing_dataset(q, dataset), guess), dataset)
	return acc/k

\end{lstlisting}



\begin{lstlisting}[label = code-p2-2a, language=Python, caption=Python Code for Problem 2 - 2 - (a)]
import numpy as np
import matplotlib.pyplot as plt
import random
#GENERATING DATA SIZE AND CONRRESPONDING PARAMETER
def gen_data(d, n): return np.array([[random.choice([-1,1]) for _ in range(d)] for i in range(n)])

def F_test(A, X): return np.dot(A, X.transpose())

def query_avg(data):
	return np.array(map(sum, data.transpose()))/(len(data)*1.0)

def gaussi_mech(query, sigma, data): return query(data) + np.array([random.gauss(0, sigma) for _ in data[0]]) 
def scores(A, data): return F_test(A, data)

def true_positive(sc1, sc2):
	return sum(map(lambda s: 1.0 if s*1.0/len(sc1) > 0.95 else 0, 
		[sum(map(lambda s2: 1.0 if s1 > s2 else 0, sc2)) for s1 in sc1])) / len(sc1)

def exprmt(d, n, sigma):
	p, k = 0.0, 1
	for i in range(k):
		data, out_data = gen_data(d, n), gen_data(d, n)
		A = gaussi_mech(query_avg, sigma, data)
		p += true_positive(scores(A, data), scores(A, out_data))
	return p/k

def exprmt_d(n, d_list, sigma): return [exprmt(d, n, sigma) for d in d_list]

def plot_accuracy(ys, ns):
	plt.figure()
	plt.plot(ns, ys, "ro-")
	plt.xlabel("d / dimension of the database (# of attributes)")
	plt.ylabel("true positive rate")
	plt.legend()
	plt.grid()
	plt.show()

if __name__ == "__main__":
	d_list = [100, 200, 400, 800, 2000, 5000]
	tp = exprmt_d(100, d_list, 1/3.0)
	plot_accuracy(tp, d_list)

\end{lstlisting}


\begin{lstlisting}[label = code-p2-2b, language=Python, caption=Python Code for Problem 2 - 2 - (b)]
def rounding_mech(query, sigma, data):
	return np.array(map(round, query(data)/sigma)) * sigma
\end{lstlisting}


\begin{lstlisting}[label = code-p2-3, language=Python, caption=Python Code for Problem 2 - 3]
import numpy as np
import matplotlib.pyplot as plt
import random
import math
from functools import partial

#GENERATING DATA SIZE AND CONRRESPONDING PARAMETER
def gen_data(d, n):
	return np.array([[random.choice([-1,1]) for _ in range(d)] for i in range(n)])

def gen_cfft(d):
	return np.array([random.choice([-1/math.sqrt(d), 1/math.sqrt(d)]) for _ in range(d)])

def gen_label(data, c, sigma):
	return np.dot(c, data.transpose()) + np.array([random.gauss(0, sigma) for _ in data]) 

def LSLR(data, label):
	return np.linalg.lstsq(data, label)[0]

def classifier(w, target):
	return 1.0 if abs(np.dot(w, target[0].transpose()) - target[1]) < 0.01 else 0.0

def scores(w, pos_targets, neg_targets):
	return [classifier(w, p) for p in pos_targets], [classifier(w, p) for p in neg_targets]

def true_positive(sc1, sc2):
	return sum(sc1) / (len(sc1))

def false_negtive(sc1, sc2):
	return 1.0 - sum(sc1) / (len(sc1))

def true_negtive(sc1, sc2):
	return sum(sc2) / (len(sc2))

def false_positive(sc1, sc2):
	return 1.0 - sum(sc2) / (len(sc2))

def exprmt(d, n, sigma):
	data, coefft, neg_data = gen_data(d, n), gen_cfft(d), gen_data(d, n)
	y, neg_y = gen_label(data, coefft, sigma), gen_label(neg_data, coefft, sigma)
	score = scores(LSLR(data, y), zip(data, y), zip(neg_data, neg_y))
	return true_positive(score[0], score[1]), true_negtive(score[0], score[1])

def exprmt_d(n, d_list, sigma):
	return [exprmt(d, n, sigma) for d in d_list]

def plot_accuracy(ys, ns):
	plt.figure()
	plt.plot(ns, [y[0] for y in ys], "ro-", label = "true positive rate")
	plt.plot(ns, [y[1] for y in ys], "bo-", label = "true negtive rate")
	plt.xlabel("d / dimension of the database (# of attributes)")
	plt.ylabel("accuracy rate")
	plt.legend()
	plt.grid()
	plt.show()

if __name__ == "__main__":
	d_list = [100, 200, 400, 800, 2000, 3000, 5000]
	tp = exprmt_d(100, d_list, 0.1)
	plot_accuracy(tp, d_list)
\end{lstlisting}

%
\begin{lstlisting}[label = code-p4-3, language=Python, caption=Python Code for Problem 4 - 3]
import numpy as np
import matplotlib.pyplot as plt
import random

#GENERATING DATA SIZE AND CONRRESPONDING PARAMETER
def gen_data(d, n,  M, sigma):
	unnormx = [random.choice(M) + np.array([random.gauss(0, sigma) for _ in range(d)]) for i in range(n)]
	return np.array([ x / sum(abs(x)) if sum(abs(x)) > 1 else x for x in unnormx]) 

def gen_point(d):
	p = np.array([random.uniform(-1, 1) for _ in range(d)])
	while sum(abs(p)) > 1.0:
		p = np.array([random.uniform(-1, 1) for _ in range(d)])
	return np.array(p)

#PURE K_MEAN ALGORITHM
def k_clustering(data, T, eps, k):
	ctrs = np.array([gen_point(len(data[0])) for _ in range(k)]) #initialize the centers
	ctrs_record = []
	for _ in range(T):
		distance, S, used = [[sum(abs(row - c)) for c in ctrs] for row in data], [[] for _ in range(k)], [False]*len(data)
		for j in range(k):
			for i in range(len(data)):
				if distance[i][j] <= min(distance[i]) and not used[i]:
					used[i] = True
					S[j].append(data[i])
		ctrs = [sum(S[i])/ len(S[i]) if S[i] != [] else gen_point(len(data[0])) for i in range(k)]
		ctrs_record.append(np.array(ctrs))
	return np.array(ctrs), np.array(ctrs_record)

#PRIVATE VERSION OF K_MEAN ALGORITHM
def k_clustering_private(data, T, eps, k):
	ctrs = np.array([gen_point(len(data[0])) for _ in range(k)]) #initialize the centers
	eps, ctrs_record = eps/(2.0 * T), []
	for _ in range(T):
		distance, S, used = [[sum(abs(row - c)) for c in ctrs] for row in data], [[] for _ in range(k)], [False]*len(data)
		for j in range(k):
			for i in range(len(data)):
				if distance[i][j] <= min(distance[i]) and not used[i]:
					S[j].append(data[i])
					used[i] = True # prevent two data point added to two different sets when they have the same distance
		ctrs = [(sum(S[i]) + np.random.laplace(0.0, eps))/ (len(S[i]) + np.random.laplace(0.0, 2*eps)) if len(S[i]) > 5 else gen_point(len(data[0])) for i in range(k)]
		ctrs_record.append(np.array(ctrs))
	return np.array(ctrs), np.array(ctrs_record)


def plot_accuracy(data, centers, M):
	def obx (data):
		return list(data.transpose()[0])
	def oby (data):
		return list(data.transpose()[1])
	def ithcx(data, i):
		return data.transpose()[0][i]
	def ithcy(data, i):
		return data.transpose()[1][i]
	plt.figure()
	plt.plot(obx(data), oby(data), "o", color="orchid", label="data")
	for i in range(len(M)):
		plt.plot((ithcx(centers, i)), (ithcy(centers, i)), "*--", label="centers" + str(i))
	plt.plot(obx(M), oby(M), "g^", label="M")
	plt.legend()
	plt.grid()
	plt.show()




if __name__ == "__main__":

	#SETTING UP THE PARAMETERS WHEN DOING GROUPS EXPERIMENTS
	M = np.array([[0, 0.5], [0.2, -0.2], [-0.2, -0.2]])
	data = gen_data(2, 2000, M, 0.01)
	#EXPERIMENTS WIEH PRIVATE K_MEANS
	centers, records = k_clustering_private(data, 10, 0.1, 3)
	plot_accuracy(data, records, M)
	#EXPERIMENTS WIEH PURE K_MEANS
	centers, records = k_clustering(data, 10, 0.1, 3)
	plot_accuracy(data, records, M)



\end{lstlisting}



\end{document}
