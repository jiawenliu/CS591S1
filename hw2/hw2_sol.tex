\documentclass[11pt]{article}
\usepackage[margin=1in,footskip=0.25in]{geometry}
\usepackage{hyperref}
\usepackage{graphicx}
\usepackage[T1]{fontenc}
\usepackage{listings}
\usepackage{xcolor}
 
\definecolor{codegreen}{rgb}{0,0.6,0}
\definecolor{codegray}{rgb}{0.5,0.5,0.5}
\definecolor{codepurple}{rgb}{0.58,0,0.82}
\definecolor{backcolour}{rgb}{0.95,0.95,0.92}
 
\lstdefinestyle{mystyle}{
    backgroundcolor=\color{backcolour},   
    commentstyle=\color{codegreen},
    keywordstyle=\color{magenta},
    numberstyle=\tiny\color{codegray},
    stringstyle=\color{codepurple},
    basicstyle=\ttfamily\footnotesize,
    breakatwhitespace=false,         
    breaklines=true,                 
    captionpos=b,                    
    keepspaces=true,                 
    numbers=left,                    
    numbersep=5pt,                  
    showspaces=false,                
    showstringspaces=false,
    showtabs=false,                  
    tabsize=2
}
 
\lstset{style=mystyle}

\usepackage{amsthm}
\usepackage{amsmath}
\usepackage{graphicx}
\usepackage{multicol, latexsym, amssymb}
\usepackage{blindtext}
\usepackage{subcaption}
\usepackage{caption}
\usepackage{algorithm}
\usepackage{algorithmic}

\usepackage{tabu}

\begin{document}

\title{
{\textbf{CS591S1 Homework 2: More on Differential Privacy}}
}
\author{Jiawen Liu\\
Collaborators: none.}

\date{}
\maketitle

\section{Problem 1}

\begin{enumerate}
	\item (Sensitivity)

	Given data set $\data$, for arbitrary $y\in \{ 1, \cdots, R \}$ and adjacent dataset $\data'$ by insertion or deletion of one data point $x_k$, we have following cases for insertion (deletion will be symmetric):
	\begin{itemize}
		\item $y = x_k$:
		\[
		q(y; \data') 
		= - | \sum_{i = 0}^{n} sign(y - x_i) + sign( y - x_k) | 
		= - | \sum_{i = 0}^{n} sign(y - x_i) + 0 |
		= - | \sum_{i = 0}^{n} sign(y - x_i) |
		= q(y; \data)
		\]

		\item $y < x_k$
		%
		\[
		\begin{array}{rcl}
		q(y; \data') 
		& = &
		- | \sum_{i = 0}^{n} sign(y - x_i) + sign( y - x_k) | \\
		& = &
		- | \sum_{i = 0}^{n} sign(y - x_i) - 1 | ~ (\star)
		\end{array}
		\]
		%
		By triangle inequality, we have:
		\[
		\begin{array}{rcl}
		(\star) & 
		 \leq & 
		 - | \sum_{i = 0}^{n} sign(y - x_i) |  + 1
		= q(y; \data) + 1\\
		%
		(\star) &
		 \geq & 
		 - | \sum_{i = 0}^{n} sign(y - x_i) |  - 1
		 =
		 q(y; \data) - 1
		\end{array}
		\]
		%
		Then we can get:
		%
		\[
		-1 \leq q(y; \data') - q(y; \data) \leq 1
		\]
		%
		\item $y > x_k$
		%
		\[
		\begin{array}{rcl}
		q(y; \data') 
		& = &
		- | \sum_{i = 0}^{n} sign(y - x_i) + sign( y - x_k) | \\
		& = &
		- | \sum_{i = 0}^{n} sign(y - x_i) + 1 | ~ (\star)
		\end{array}
		\]
		%
		By triangle inequality, we have:
		\[
		\begin{array}{rcl}
		(\star) & 
		 \leq & 
		 - | \sum_{i = 0}^{n} sign(y - x_i) |  + 1
		= q(y; \data) + 1\\
		%
		(\star) &
		 \geq & 
		 - | \sum_{i = 0}^{n} sign(y - x_i) |  - 1
		 =
		 q(y; \data) - 1
		\end{array}
		\]
		%
		Then we can get:
		%
		\[
		-1 \leq q(y; \data') - q(y; \data) \leq 1
		\]
		%
	\end{itemize}
	The Deletion is symmetric where we can get: $-1 \leq q(y; \data) - q(y; \data') \leq 1$ in the same way.
	%
	\\
	%
	Then, we can conclude from all cases, the $| q(y; \data) - q(y; \data') | \leq 1$, i.e., the sensitivity be at most 1.
	%
	%
	\item 
	\begin{proof}
	By the definition of $rank_{\data}(y)$, we have:
	\[
	|rank_{\data}(y) - \frac{n}{2}| = - q(y; \data).
	\]
	Then, we know:
	%
	\[
	\begin{array}{lll}
	Pr_{y \sim A_{\epsilon}(\data)}[|rank_{\data}(y) - \frac{n}{2}|
	> 
	c \cdot \frac{\ln(R) + \ln(1 / \beta)}{\epsilon}] 
	& \equiv &
	Pr_{y \sim A_{\epsilon}(\data)}[- q(y; \data)
	> 
	c \cdot \frac{\ln(R) + \ln(1 / \beta)}{\epsilon}]\\
	& = &
	Pr_{y \sim A_{\epsilon}(\data)}[q(y; \data)
	\leq 
	- c \cdot \frac{\ln(R) + \ln(1 / \beta)}{\epsilon}]	
	\end{array}
	\]
	%
	By definition of exponential mechanism, we have:
	%
	\[
	\begin{array}{lll}
	Pr_{y \sim A_{\epsilon}(\data)}[q(y; \data)
	\leq 
	- c \cdot \frac{\ln(R) + \ln(1 / \beta)}{\epsilon}]
	& = &
	\sum\limits_{y| q(y; \data) < - c \cdot \frac{\ln(R) + \ln(1 / \beta)}{\epsilon}}
	\frac{\exp \big( q(y; \data)\epsilon / 2S \big)}
	{\sum\limits_{y}\exp \big( q(y; \data)\epsilon / 2S \big)}\\
	%
	& \leq &
	R
	\frac{\exp \big( - c \cdot \frac{\ln(R) + \ln(1 / \beta)}{\epsilon}
	\frac{\epsilon}{2S} \big)}
	{\sum\limits_{y}\exp \big( q(y; \data)\epsilon / 2S \big)}\\	
	%
	& = &
	R
	\frac{\exp \big(  \frac{c}{2} (\ln(\frac{1}{R}) + \ln(\beta)) \big)}
	{\sum\limits_{y}\exp \big( q(y; \data)\epsilon / 2S \big)} ~(\star)
	\end{array}
	\]
	%
	Since the only one optimal output candidate is the median value where $q(y, \data) = 0$, so we have:
	%
	\[
	\begin{array}{lll}
	(\star) & \leq &
	\frac{\exp \big(  \frac{c}{2} (\ln(\frac{1}{R}) + \ln(\beta)) \big)}
	{\exp ( 0)}\\	
	\end{array}
	\]
	%
	In order to have this probability be at most $\beta$, we take the equality and get:
	%
	\[
	\begin{array}{rcl}
	\frac{\exp \big(  \frac{c}{2} (\ln(\frac{1}{R}) + \ln(\beta)) \big)}
	{\exp ( 0)}
	& = & \beta\\
	R^{1 - \frac{c}{2}}	& = & \beta^{1 - \frac{c}{2}}
	\end{array}
	\]
	%
	Since we have $R \geq 1$ and $\beta \in (0,1)$, there exists $c = 2$ which can make the equation holds.
	%
	%
	\end{proof}
\end{enumerate}


\section{Problem 2}
\begin{proof}
The proof are developed by two cases where the insertion and deletion whether or not make changes on empty bins:
\begin{itemize}
\caseL{changes on nonempty bins}
Taking arbitrary data set $\data$ and one possible output $S$, there are two cases of insertion to create the adjacent data set $\data'$:
%
\begin{itemize} 
\subcaseL{insertion} Inserting one data that makes an empty bin ($k \in \domain$) be nonempty, we have the 
%
\[
	\begin{array}{rcl}
	\frac{Pr[A(\data') = S]}
	{Pr[A(\data) = S]}
	& = & 
	\frac{ Pr[A(\data) = S \land \tilde{c'_k} < \tau ] }
	{Pr[A(\data) = S]}\\
	& = &
	\frac{Pr[A(\data) = S] \cdot Pr[ \tilde{c'_k} < \tau ] }
	{Pr[A(\data) = S]}\\
	& = &
	Pr[ 1 + Lap(\frac{1}{\epsilon}) < \tau]\\
	& \leq &
	(1 - \frac{\delta}{2})
	\end{array}
\]

\subcaseL{deletion.} Deleting one data point that makes the existing nonempty bins ($k \in \domain$) be empty bins, we have the probability as:
%
\[
	\begin{array}{rcl}
	\frac{Pr[A(\data') = S]}
	{Pr[A(\data) = S]}
	& = & 
	\frac{ Pr[A(\data') = S] }
	{Pr[A(\data') = S ~ \land ~ \tilde{c_k} < \tau ]}\\
	& = &
	\frac{Pr[A(\data') = S]] }
	{Pr[A(\data') = S] \cdot Pr[ \tilde{c_k} < \tau ]}\\
	& = &
	\frac{1}{Pr[ 1 + Lap(\frac{1}{\epsilon}) < \tau]}\\
	& \geq &
	\frac{1}{1 - \frac{\delta}{2}}
	\end{array}
\]

\end{itemize}

\caseL{no change on nonempty bins}
Taking arbitrary data set $\data$ and a possible output $S$, There are two cases of creating the adjacent data set $\data'$:
%
\begin{itemize} 
\subcaseL{insertion} insert one data which makes an empty bin ($k \in \domain$) be nonempty, we have the 
\[
	\begin{array}{rcl}
	\frac{Pr[A(\data') = S]}
	{Pr[A(\data) = S]}
	& = & 
	\frac{Pr[A(\data \setminus k) = S\setminus k 
	\land \tilde{c'_k} = \tilde{c_k}]}
	{Pr[A(\data) = S]}\\
	& = &
	\frac{Pr[A(\data \setminus k) = S\setminus k ] \cdot Pr[\tilde{c'_k} = \tilde{c_k}]}
	{Pr[A(\data) = S]}\\
	& = &
	\frac{Pr[ c'_k + Lap(\frac{1}{\epsilon}) = \tilde{c_k}]}
	{Pr[c_k + Lap(\frac{1}{\epsilon}) = \tilde{c_k}]}\\
	& \leq &
	e^{\epsilon}
	\end{array}
\]
%
\subcaseL{deletion} insert one data point in the existing nonempty bins ($k \in \domain$).
We have the probability as:
\[
	\begin{array}{rcl}
	\frac{Pr[A(\data') = S]}
	{Pr[A(\data) = S]}
	& = & 
	\frac{Pr[A(\data \setminus k) = S\setminus k 
	\land \tilde{c'_k} = \tilde{c_k}]}
	{Pr[A(\data) = S]}\\
	& = &
	\frac{Pr[A(\data \setminus k) = S\setminus k ] \cdot Pr[\tilde{c'_k} = \tilde{c_k}]}
	{Pr[A(\data) = S]}\\
	& = &
	\frac{Pr[ c'_k + Lap(\frac{1}{\epsilon}) = \tilde{c_k}]}
	{Pr[c_k + Lap(\frac{1}{\epsilon}) = \tilde{c_k}]}\\
	& \geq &
	e^{-\epsilon}
	\end{array}
\]

\end{itemize}

\end{itemize}
By summarization, the probability of failure would be $\delta$ above cases. So we have the algorithm be $(\epsilon, \delta)$-DP. 
\end{proof}



\end{document}
