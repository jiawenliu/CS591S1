\documentclass[11pt]{article}
\usepackage[margin=1in,footskip=0.25in]{geometry}
\usepackage{hyperref}
\usepackage{graphicx}
\usepackage[T1]{fontenc}
\usepackage{listings}
\usepackage{xcolor}
 
\definecolor{codegreen}{rgb}{0,0.6,0}
\definecolor{codegray}{rgb}{0.5,0.5,0.5}
\definecolor{codepurple}{rgb}{0.58,0,0.82}
\definecolor{backcolour}{rgb}{0.95,0.95,0.92}
 
\lstdefinestyle{mystyle}{
    backgroundcolor=\color{backcolour},   
    commentstyle=\color{codegreen},
    keywordstyle=\color{magenta},
    numberstyle=\tiny\color{codegray},
    stringstyle=\color{codepurple},
    basicstyle=\ttfamily\footnotesize,
    breakatwhitespace=false,         
    breaklines=true,                 
    captionpos=b,                    
    keepspaces=true,                 
    numbers=left,                    
    numbersep=5pt,                  
    showspaces=false,                
    showstringspaces=false,
    showtabs=false,                  
    tabsize=2
}
 
\lstset{style=mystyle}

\usepackage{mathtools}

\DeclarePairedDelimiter\abs{\lvert}{\rvert}%
\DeclarePairedDelimiter\norm{\lVert}{\rVert}%

% Swap the definition of \abs* and \norm*, so that \abs
% and \norm resizes the size of the brackets, and the 
% starred version does not.
\makeatletter
\let\oldabs\abs
\def\abs{\@ifstar{\oldabs}{\oldabs*}}
%
\let\oldnorm\norm
\def\norm{\@ifstar{\oldnorm}{\oldnorm*}}
\makeatother

\newcommand*{\Value}{\frac{1}{2}x^2}

\newcommand{\caseL}[1]{\item[\textbf{case}] \textbf{#1}\newline}
\newcommand{\subcaseL}[1]{\item[\textbf{subcase}] \textbf{#1}\newline}

\newcommand{\todo}[1]{{\footnotesize \color{red}\textbf{[[ #1 ]]}}}


\newcommand{\data}{\mathbf{x}}

\newcommand{\datay}{\mathbf{y}}
\newcommand{\domain}{\mathcal{X}}
\newcommand{\domainy}{\mathcal{Y}}


\usepackage{amsthm}
\usepackage{amsmath}
\usepackage{graphicx}
\usepackage{multicol, latexsym, amssymb}
\usepackage{blindtext}
\usepackage{subcaption}
\usepackage{caption}
\usepackage{algorithm}
\usepackage{algorithmic}

\usepackage{tabu}

\begin{document}

\title{
{\textbf{CS591S1 Homework 3: Algorithms for Differential Privacy}}
}
\author{Jiawen Liu\\
Collaborators: none.}

\date{}
\maketitle

\section{Answering Many Queries with the Exponential Mechanism}
\begin{enumerate}
\item 
\begin{proof}
By the tail bound, we can get:
\[
	Pr_{Y \sim N(0, \sigma)}[Y > t] 
	\leq 
	\frac{\sigma}{\sqrt{2\pi}} e^{- t^2 / 2\sigma^2}
\]
Then we can get:
\[
\begin{array}{rcl}
	Pr[|err| < \alpha] 
	& \geq & 1 - 
	\frac{\sigma}{\sqrt{2\pi}} e^{- \alpha^2 / 2\sigma^2}\\
	& \geq & 1 - 
	e^{- \alpha^2 / 2\sigma^2}
\end{array}
\]
To guarantee $1 - e^{- \alpha^2 / 2\sigma^2}$ be a constant, it is sufficient to guarantee $- \alpha^2 / 2\sigma^2 = c$.
\\
Since $\sigma = O(\frac{\sqrt{d}}{n} \cdot \frac{\log(1/\delta)}{\epsilon})$, so we can have:
\\
$O(- \alpha^2 n^2 \epsilon^2 / 2 d \log(1/\delta)) \leq c$ 
and then, $n \geq \Omega\big( \frac{\sqrt{d}\log(1/\delta)}{\epsilon \alpha} \big)$
\end{proof}
%
%
%
\item 
\begin{proof}
To guarantee there exist $\datay$ that $disc(\datay; \data) \leq \alpha$, it is sufficient to show:
\[
	Pr[disc(\datay; \data) \leq \alpha] \geq 0 = 1 - 1
\]
Let $f_j(\data) = \sum\limits_{i = 1}^{n} f_j(x_i)$, $f_j(\datay) = \sum_{i = 1}{n} f_j(y_i)$ and $\mu_j = \frac{1}{n} \sum\limits_{i = 1}^{n} f_j(x_i)$, by the Chernoff bound and union bound, we have following:
%
\[
\begin{array}{rcl}
	Pr[disc(\datay; \data) \leq \alpha]
	& = &
	Pr[\max\limits_{j \in [d]}\abs{\mu_j - \frac{1}{k}f_j(\datay)} \leq \alpha]\\
	& = &
	Pr[\abs{\mu_1 - \frac{1}{k}f_1(\datay)} \leq \alpha 
	\land \cdots \land  \abs{\mu_d - \frac{1}{k}f_d(\datay)} \leq \alpha ]\\
	& = &
	1 - Pr[\abs{\mu_1 - \frac{1}{k}f_1(\datay)} > \alpha 
	\lor \cdots \lor  \abs{\mu_d - \frac{1}{k}f_d(\datay)} > \alpha ]\\
	& \geq &
	1 - \sum_{j = 1}^{d}
	Pr[\abs{\mu_j - \frac{1}{k}f_j(\datay)} > \alpha]
		%
		~~~~(\text{applying the union bound})\\
		%
		& = & 1 - \sum\limits_{j = 1}^{d} 
		Pr[|f_j(\datay) - k \mu_j| > k \alpha]\\
		& \geq & 1 - 2 \sum\limits_{i = 1}^{d}
		\exp(- \frac{\alpha^2 / \mu_i}{ 2 + \alpha / \mu_i}
		k \mu_i) 
		%
		~~~~(\text{applying the Chernoff bound})\\
		%
		& = & 1 - 2 \sum\limits_{i = 1}^{d}
		\exp(- \frac{k \alpha^2}{ 2\mu_i + \alpha })\\
		%		
		& \geq & 1 - 2d\exp(- \frac{k \alpha^2}{ 2 + \alpha })
\end{array}
\] 
To guarantee $1 - 2d\exp(- \frac{k \alpha^2}{ 2 + \alpha }) > 0$, we have: $k < (2 + \alpha) \ln 2d / \alpha^2 = O(\ln d / \alpha^2)$
\end{proof}
%
%
%
\item 
\begin{proof}
Let $\data, \data' \in \domain^n$ be arbitrary adjacent data set, we have for any $\datay \in \domainy$:
 $$
 \begin{array}{rcl}
 |q(\datay; \data) - q(\datay; \data')| 
 & = & \abs{
 \max\limits_{j \in [d]}
  \abs{\frac{1}{n}\sum_{i = 1}^{n}f_j(x_i) - \frac{1}{k}\sum_{i = 1}^{k}f_j(y_i)}
  - \max\limits_{j \in [d]}
  \abs{\frac{1}{n}\sum_{i = 1}^{n}f_j(x'_i) - \frac{1}{k}\sum_{i = 1}^{k}f_j(y_i)}}
 \\
 & \leq &
 \abs{
 \max\limits_{j \in [d]}
  (\abs{\frac{1}{n}\sum_{i = 1}^{n}f_j(x_i) - \frac{1}{k}\sum_{i = 1}^{k}f_j(y_i)}
  -
  \abs{\frac{1}{n}\sum_{i = 1}^{n}f_j(x'_i) - \frac{1}{k}\sum_{i = 1}^{k}f_j(y_i)})
  }
 \\
 & \leq &
 \abs{
 \max\limits_{j \in [d]}
  (\abs{\frac{1}{n}\sum_{i = 1}^{n}f_j(x_i) - \frac{1}{k}\sum_{i = 1}^{k}f_j(y_i)
  -
  \frac{1}{n}\sum_{i = 1}^{n}f_j(x'_i) - \frac{1}{k}\sum_{i = 1}^{k}f_j(y_i)})
  }
  \\
 & = &
 \frac{1}{n}\abs{
 \max\limits_{j \in [d]}
  \abs{\sum_{i = 1}^{n}f_j(x_i)
  -
  \sum_{i = 1}^{n}f_j(x'_i)
  }
  } ~ (\star)
  \end{array}
 $$
 By sensitivity of $f_j$ is $1$ for all $j$, we have $\star \leq \frac{1}{n}$.
\end{proof}
%
%
%
\item 
\begin{proof}
By the definition of exponential mechanism, we have:
\[
\begin{array}{rcl}
	Pr[disc(\datay; \data) \leq c] 
	& = & 
	1 - \sum_{disc(\datay; x) > c}\frac{\exp{(- \epsilon n \cdot disc(\datay; x)/2)}}
	{\sum_{\datay' \in \domainy} \exp{(- \epsilon n \cdot disc(\datay'; x)/2)}} \\
	& \geq &
	1 - \frac{\abs{\domainy}\exp({-\epsilon n c/2} )}
	{ \exp({-\epsilon n \alpha^*/2})} \\
	& = & 
	1 - \abs{\domain}^k
	\exp\big(\frac{-\epsilon n (c - \alpha^*)}{2} \big) \\	
\end{array}
\]
Let $\beta = \abs{\domain}^k\exp\big(\frac{-\epsilon n (c - \alpha^*)}{2} \big)$, we can solve: $c = \alpha^* + 2 \frac{k\ln(\abs{\domain}) - \ln(\beta)}{\epsilon n}$, i.e.:
\[
	Pr[disc(\datay; \data) \leq
	\alpha^* + 2 \frac{k\ln(\abs{\domain}) - \ln(\beta)}{\epsilon n}]
	\geq 1 - \beta
\]
\end{proof}
%
%
%
\item
%
\begin{proof}
By the last one problem, we have:
\[
	Pr[disc(\datay; \data) \leq \alpha] 
	\geq 
	1 - \abs{\domain}^k
	\exp\big(\frac{-\epsilon n (\alpha - \alpha^*)}{2} \big) 
\]
In order to guarantee this probability is bounded by $1 - \beta$, it is sufficient to set 
$\abs{\domain}^k \exp\big(\frac{-\epsilon n (\alpha - \alpha^*)}{2} \big) = \beta$.
Then we solve for $n$ and get 
$n \geq \frac{2\ln 1/\beta + k \ln |\domain|}{\epsilon(\alpha)}$.
\\
Given $k = O(4\log d/\alpha^2)$, we have: 
$n \geq \frac{2\ln 1/\beta + 4\log d \ln |\domain|/\alpha^2}{\epsilon \alpha} 
= O(\frac{2\ln 1/\beta}{\epsilon \alpha} + \frac{4\log d \ln |\domain|}{\epsilon \alpha^3})$.
%
%
\end{proof}
%
%
\item 
By straightforward implementation, we have:
\\
(1). Assume compute $f_j(x_i)$ takes $\abs{\domain}$ time, then we know compute $f_j(x_i)$ for $x_i \in \data$ and $j \in [d]$ take $O(nd\abs{\domain})$ time.
\\
(2). We can also know compute $f_j(y_i)$ for $y_i \in \datay$ and $j \in [d]$ take $O(kd\abs{\domain})$ time.
\\
So compute one $q(\datay; \data)$ takes $O(kd\abs{\domain} + nd\abs{\domain})$ time.
\\
(3). To compute $q(\datay; \data)$ for $\datay \in \domainy$, we have the computation time be $O(\abs{\domain}^{k} (kd\abs{\domain} + nd\abs{\domain}))$.
\\
Since we can pre-compute $f_j(x_i)$ to save time, so the running time is: $O(\abs{\domain}^{k + 1} kd + nd\abs{\domain} )$.
\\
(4). Assume normalization of $\datay$ takes O(1) time, so normalization process takes $O(\abs{\domain} )$ time.
\\
(5). Assume sampling takes O(1) time.
\\
We get the total running time be:
$O(\abs{\domain}^{k + 1} kd + nd\abs{\domain} + \abs{\domain} + 1 )$.
%
%
%
%
\item 
%
For Gaussian mechanism, we have $n \geq \Omega\big( \frac{\sqrt{d}\log(1/\delta)}{\epsilon \alpha} \big) 
= \Omega\big( \frac{\sqrt{8m^3}\log(1/\delta)}{\epsilon \alpha} \big)$.
\\
For Exponential mechanism, we have $n \geq O\big(\frac{2\ln 1/\beta}{\epsilon \alpha} + \frac{4\log d \ln |\domain|}{\epsilon \alpha^3}\big)
=
O(\frac{2\ln 1/\beta}{\epsilon \alpha} + \frac{8\log 2m^3 \ln |\domain|}{\epsilon \alpha^3})$.

Since $O(\log d) < O(\sqrt{d})$ asymptotically, we know exponential mechanism scales better in $d$ asymptotically.
\end{enumerate}
%
%
%

\clearpage
\section{Implementing CDF Estimation}
\begin{enumerate}
	\item 
	The \textbf{code documentation} is attached in Appendix \ref{code-alg2}.
	\item
	The \textbf{experimental results} is plotted in Figure \ref{fig-alg2} (I'm unable to get the full plot for $n = 10^5$ because it runs so slow that takes more than 1 day without any result).
		\begin{figure*}[t!]
		    \centering
		    \begin{subfigure}[t]{0.4\textwidth}
		        \centering
		        \includegraphics[width=\textwidth]{alg2-1}
		        \caption{Data size $n = 10^2$}
		    \end{subfigure}%
		    ~ 
		    \begin{subfigure}[t]{0.4\textwidth}
		        \centering
		        \includegraphics[width=\textwidth]{alg2-2}
		        \caption{Data Size $n = 10^3$}
		    \end{subfigure}
		    ~ 
		    \begin{subfigure}[t]{0.4\textwidth}
		        \centering
		        \includegraphics[width=\textwidth]{alg2-3}
		        \caption{Data Size $n = 10^4$}
		    \end{subfigure}
		    ~ 
		    \begin{subfigure}[t]{0.4\textwidth}
		        \centering
		        \includegraphics[width=\textwidth]{alg2-4}
		        \caption{Data Size $n = 10^5$}
		    \end{subfigure}
		    \caption{Algorithm 2 Simple Histogram CDF}
		    \label{fig-alg2}
		\end{figure*}
	\item
	\begin{enumerate}
		\item 
		We know $SimpleHistogramCDF$ algorithm is $\epsilon$-differentially private. 
		%
		\\
		%
		Since the $TreeHistogram$ algorithm is calling the SimpleHistogramCDF algorithm with $\epsilon / l$ for $l$ times. By the composition property of differential privacy, we have
		the $TreeHistogram$ algorithm is $\epsilon$-differentially private.

		\item 
		\begin{proof}
		When there is no error added, we have:
		$$
		Y_j = \frac{\#\{i : x_i \in  [2^{-l} (j - 1), 2^{-l} j]\}}{n},
		$$
		Since $CDF_{\data}(t) = \frac{\#\{i : x_i < t\}}{n}$ and we know $t$ is a multiple of $2^{-l}$. Then there must exist $k$ s.t. $t = k 2^{-l}$. Then we have:
		$$
		CDF_{\data}(t) = \frac{\#\{i : x_i < k 2^{-l}\}}{n} 
		= \sum_{j = 1}^{k} \frac{\#\{i : (j - 1) 2^{-l} \leq x_i < j 2^{-l}\}}{n}
		= \sum_{j = 1}^{k} Y_j.
		$$
		i.e., we can accurately reconstruct the $CDF_{\data}(t)$ by sum of $Y_j$.

		\end{proof}
		\item

		Define the $\hat{CDF}(t) = \frac{1}{l}\sum\limits_{\alpha = 1/2}^{1/2^l}\hat{CDF}_{\alpha}(t)$.

		% \begin{algorithm}
		% \caption{CDF}
		% \label{alg_p1-1-2}
		% \begin{algorithmic}
		% \REQUIRE The observed results $a$ from query.
		% \STATE {\bf Initialize vector s: $s[i] = a[i] - a[i - 1]$} 
		% \COMMENT {as the reconstructed dataset.} 
		% \STATE  {\bf for}\ $i\in [a.length]$\ {\bf do}.  
		% \STATE \qquad {\bf If} $s > 1$  {\bf do}.
		%     $s[i] = 1$
		% \STATE \qquad {\bf Elif} $s < 0$  {\bf do}.
		% 	$s[i] = 0$
		% \RETURN $s$.
		% \end{algorithmic}
		% \end{algorithm}

	\end{enumerate}
	\item
	The \textbf{code documentation} is attached in Appendix \ref{code-alg3}
	
	The \textbf{experimental results} is plotted in Figure \ref{fig-alg3} (I'm unable to get the plot for $n = 10^5$ because of the huge running time).
		\begin{figure*}[t!]
		    \centering
		    \begin{subfigure}[t]{0.4\textwidth}
		        \centering
		        \includegraphics[width=\textwidth]{alg3-1}
		        \caption{Data size $n = 10^2$}
		    \end{subfigure}%
		    ~ 
		    \begin{subfigure}[t]{0.4\textwidth}
		        \centering
		        \includegraphics[width=\textwidth]{alg3-2}
		        \caption{Data Size $n = 10^3$}
		    \end{subfigure}
		    \caption{Algorithm 3 Tree Histogram CDF}
		    \label{fig-alg3}
		\end{figure*}
\end{enumerate}

\newpage
\section*{Appendix}

\begin{lstlisting}[label=code-p1-1, language=Python, caption=Python Code For Problem 1 - 1, Attack without Side Information]
import random
import numpy as np

#GENERATING DATA SIZE AND CONRRESPONDING PARAMETER
def gen_dataset(n):
	return [random.randint(0, 1)for i in range(n)]
def gen_datasizes(r, step):
	return [i*step for i in range(r[0]/step,r[1]/step + 1)]

#RELEASING THE NOIZED VERSION OF DATABASE
def releasing_dataset(dataset):
	return [sum(dataset[:(i + 1)]) + random.randint(0,1) for i in range(len(dataset))]

#ATTACK WITH ONLY THE KNOWLEDGE OF THE OBSERVATION OF ONE DATABASE
def attack_no_aux(observation):
	rec_counter = [observation[0] - 1 if observation[0] > 1 else observation[0]]
	for i in range(1, len(observation)):
		s = observation[i] - rec_counter[i - 1]
		if s > 1:
			rec_counter.append(observation[i] - 1)
		elif s < 0:
			rec_counter.append(observation[i])
			j = i
			while j - 1 >= 0 and rec_counter[j] < rec_counter[j - 1]:
				j -= 1
				rec_counter[j] -= 1
		else:
			rec_counter.append(observation[i])
	return np.array([rec_counter[0]] + [rec_counter[i] - rec_counter[i - 1] for i in range(1, len(observation))])

###ATTACK OF MINIMIZING ERROR WITH ONLY THE KNOWLEDGE OF OBSERVATION
def attack_no_aux_minerror(observation):
	n, error, r = len(observation), float("inf"), observation
	for i in range(1000):
		s = gen_dataset(n)
		e = sum(abs(releasing_dataset(gen_query(n), s) - observation))
		if e < error:
			r = s
			error = e
	return r

###CALCULATING THE ACCURACY
def accuracy(att, data):
	return sum([1 if att[i]==data[i] else 0 for i in range(len(att))])/(len(att)*1.0)

def plot_accuracy(ys, ns):
	plt.figure()
	plt.plot(ns, ys, "ro-", label = "Accuracy.")
	plt.plot(ns,[sum(ys)/len(ys)]*len(ys), "b-", label="Average Acc.",linewidth=3.0)
	plt.xlabel("n / size of the database")
	plt.ylabel("accuracy / fraction of the bits recovered")
	plt.title("Linear Attack")
	plt.legend()
	plt.grid()
	plt.show()

#EXPERIMENTING WITH FIXED N FOR K ROUNDS
def exprmt_k(n, k):
	acc = 0.0
	for i in range(k):
		dataset, q = gen_dataset(n), gen_query(n)
		acc += accuracy(attack_no_aux(releasing_dataset(q,dataset)),dataset)
	return acc/k
def exprmt_k_ns(ns, k):
	return [testing_kround(n, k) for n in ns]

if __name__ == "__main__":
	datasizes = gen_datasizes((100,900),100)+gen_datasizes((1000,5000),200)+[50000]
	plot_accuracy(exprmt_k_ns(datasizes, 20), datasizes)

\end{lstlisting}


\begin{lstlisting}[label=code-p1-2, language=Python, caption=Python Code For Attack with side Information]
def attack_with_aux(observation, guess):
	rec_counter, n = [observation[0] - 1 if observation[0] > 1 else guess[0]], len(observation)
	certain = [-1] * n

	for i in range(1, n):
		s = observation[i] - rec_counter[i - 1]
		if s > 1:
			rec_counter.append(observation[i] - 1)
			j = i
			certain[j] = 1
			while j - 1 >= 0 and rec_counter[j - 1] < rec_counter[j] and certain[j - 1] == -1:
				j -= 1
		elif s < 0:
			rec_counter.append(observation[i])
			j = i
			certain[j] = 0
			while j - 1 >= 0 and rec_counter[j] < rec_counter[j - 1]:
				j -= 1
				rec_counter[j] -= 1
				certain[j] = 0
		else:
			rec_counter.append(observation[i])
	for i in range(n):
		if certain[i] == -1:
			certain[i] = guess[i]
	return np.array(certain)

def exprmt_k(n, k):
	acc = 0.0
	for i in range(k):
		dataset, q = gen_dataset(n), gen_query(n)
		guess = [d if random.random() >= (1.0/3) else 1 - d for d in dataset]
		acc += accuracy(attack_with_aux(releasing_dataset(q, dataset), guess), dataset)
	return acc/k

\end{lstlisting}



\begin{lstlisting}[label = code-p2-2a, language=Python, caption=Python Code for Problem 2 - 2 - (a)]
import numpy as np
import matplotlib.pyplot as plt
import random
#GENERATING DATA SIZE AND CONRRESPONDING PARAMETER
def gen_data(d, n): return np.array([[random.choice([-1,1]) for _ in range(d)] for i in range(n)])

def F_test(A, X): return np.dot(A, X.transpose())

def query_avg(data):
	return np.array(map(sum, data.transpose()))/(len(data)*1.0)

def gaussi_mech(query, sigma, data): return query(data) + np.array([random.gauss(0, sigma) for _ in data[0]]) 
def scores(A, data): return F_test(A, data)

def true_positive(sc1, sc2):
	return sum(map(lambda s: 1.0 if s*1.0/len(sc1) > 0.95 else 0, 
		[sum(map(lambda s2: 1.0 if s1 > s2 else 0, sc2)) for s1 in sc1])) / len(sc1)

def exprmt(d, n, sigma):
	p, k = 0.0, 1
	for i in range(k):
		data, out_data = gen_data(d, n), gen_data(d, n)
		A = gaussi_mech(query_avg, sigma, data)
		p += true_positive(scores(A, data), scores(A, out_data))
	return p/k

def exprmt_d(n, d_list, sigma): return [exprmt(d, n, sigma) for d in d_list]

def plot_accuracy(ys, ns):
	plt.figure()
	plt.plot(ns, ys, "ro-")
	plt.xlabel("d / dimension of the database (# of attributes)")
	plt.ylabel("true positive rate")
	plt.legend()
	plt.grid()
	plt.show()

if __name__ == "__main__":
	d_list = [100, 200, 400, 800, 2000, 5000]
	tp = exprmt_d(100, d_list, 1/3.0)
	plot_accuracy(tp, d_list)

\end{lstlisting}


\begin{lstlisting}[label = code-p2-2b, language=Python, caption=Python Code for Problem 2 - 2 - (b)]
def rounding_mech(query, sigma, data):
	return np.array(map(round, query(data)/sigma)) * sigma
\end{lstlisting}


\begin{lstlisting}[label = code-p2-3, language=Python, caption=Python Code for Problem 2 - 3]
import numpy as np
import matplotlib.pyplot as plt
import random
import math
from functools import partial

#GENERATING DATA SIZE AND CONRRESPONDING PARAMETER
def gen_data(d, n):
	return np.array([[random.choice([-1,1]) for _ in range(d)] for i in range(n)])

def gen_cfft(d):
	return np.array([random.choice([-1/math.sqrt(d), 1/math.sqrt(d)]) for _ in range(d)])

def gen_label(data, c, sigma):
	return np.dot(c, data.transpose()) + np.array([random.gauss(0, sigma) for _ in data]) 

def LSLR(data, label):
	return np.linalg.lstsq(data, label)[0]

def classifier(w, target):
	return 1.0 if abs(np.dot(w, target[0].transpose()) - target[1]) < 0.01 else 0.0

def scores(w, pos_targets, neg_targets):
	return [classifier(w, p) for p in pos_targets], [classifier(w, p) for p in neg_targets]

def true_positive(sc1, sc2):
	return sum(sc1) / (len(sc1))

def false_negtive(sc1, sc2):
	return 1.0 - sum(sc1) / (len(sc1))

def true_negtive(sc1, sc2):
	return sum(sc2) / (len(sc2))

def false_positive(sc1, sc2):
	return 1.0 - sum(sc2) / (len(sc2))

def exprmt(d, n, sigma):
	data, coefft, neg_data = gen_data(d, n), gen_cfft(d), gen_data(d, n)
	y, neg_y = gen_label(data, coefft, sigma), gen_label(neg_data, coefft, sigma)
	score = scores(LSLR(data, y), zip(data, y), zip(neg_data, neg_y))
	return true_positive(score[0], score[1]), true_negtive(score[0], score[1])

def exprmt_d(n, d_list, sigma):
	return [exprmt(d, n, sigma) for d in d_list]

def plot_accuracy(ys, ns):
	plt.figure()
	plt.plot(ns, [y[0] for y in ys], "ro-", label = "true positive rate")
	plt.plot(ns, [y[1] for y in ys], "bo-", label = "true negtive rate")
	plt.xlabel("d / dimension of the database (# of attributes)")
	plt.ylabel("accuracy rate")
	plt.legend()
	plt.grid()
	plt.show()

if __name__ == "__main__":
	d_list = [100, 200, 400, 800, 2000, 3000, 5000]
	tp = exprmt_d(100, d_list, 0.1)
	plot_accuracy(tp, d_list)
\end{lstlisting}

%
\begin{lstlisting}[label = code-p4-3, language=Python, caption=Python Code for Problem 4 - 3]
import numpy as np
import matplotlib.pyplot as plt
import random

#GENERATING DATA SIZE AND CONRRESPONDING PARAMETER
def gen_data(d, n,  M, sigma):
	unnormx = [random.choice(M) + np.array([random.gauss(0, sigma) for _ in range(d)]) for i in range(n)]
	return np.array([ x / sum(abs(x)) if sum(abs(x)) > 1 else x for x in unnormx]) 

def gen_point(d):
	p = np.array([random.uniform(-1, 1) for _ in range(d)])
	while sum(abs(p)) > 1.0:
		p = np.array([random.uniform(-1, 1) for _ in range(d)])
	return np.array(p)

#PURE K_MEAN ALGORITHM
def k_clustering(data, T, eps, k):
	ctrs = np.array([gen_point(len(data[0])) for _ in range(k)]) #initialize the centers
	ctrs_record = []
	for _ in range(T):
		distance, S, used = [[sum(abs(row - c)) for c in ctrs] for row in data], [[] for _ in range(k)], [False]*len(data)
		for j in range(k):
			for i in range(len(data)):
				if distance[i][j] <= min(distance[i]) and not used[i]:
					used[i] = True
					S[j].append(data[i])
		ctrs = [sum(S[i])/ len(S[i]) if S[i] != [] else gen_point(len(data[0])) for i in range(k)]
		ctrs_record.append(np.array(ctrs))
	return np.array(ctrs), np.array(ctrs_record)

#PRIVATE VERSION OF K_MEAN ALGORITHM
def k_clustering_private(data, T, eps, k):
	ctrs = np.array([gen_point(len(data[0])) for _ in range(k)]) #initialize the centers
	eps, ctrs_record = eps/(2.0 * T), []
	for _ in range(T):
		distance, S, used = [[sum(abs(row - c)) for c in ctrs] for row in data], [[] for _ in range(k)], [False]*len(data)
		for j in range(k):
			for i in range(len(data)):
				if distance[i][j] <= min(distance[i]) and not used[i]:
					S[j].append(data[i])
					used[i] = True # prevent two data point added to two different sets when they have the same distance
		ctrs = [(sum(S[i]) + np.random.laplace(0.0, eps))/ (len(S[i]) + np.random.laplace(0.0, 2*eps)) if len(S[i]) > 5 else gen_point(len(data[0])) for i in range(k)]
		ctrs_record.append(np.array(ctrs))
	return np.array(ctrs), np.array(ctrs_record)


def plot_accuracy(data, centers, M):
	def obx (data):
		return list(data.transpose()[0])
	def oby (data):
		return list(data.transpose()[1])
	def ithcx(data, i):
		return data.transpose()[0][i]
	def ithcy(data, i):
		return data.transpose()[1][i]
	plt.figure()
	plt.plot(obx(data), oby(data), "o", color="orchid", label="data")
	for i in range(len(M)):
		plt.plot((ithcx(centers, i)), (ithcy(centers, i)), "*--", label="centers" + str(i))
	plt.plot(obx(M), oby(M), "g^", label="M")
	plt.legend()
	plt.grid()
	plt.show()




if __name__ == "__main__":

	#SETTING UP THE PARAMETERS WHEN DOING GROUPS EXPERIMENTS
	M = np.array([[0, 0.5], [0.2, -0.2], [-0.2, -0.2]])
	data = gen_data(2, 2000, M, 0.01)
	#EXPERIMENTS WIEH PRIVATE K_MEANS
	centers, records = k_clustering_private(data, 10, 0.1, 3)
	plot_accuracy(data, records, M)
	#EXPERIMENTS WIEH PURE K_MEANS
	centers, records = k_clustering(data, 10, 0.1, 3)
	plot_accuracy(data, records, M)



\end{lstlisting}



\end{document}
